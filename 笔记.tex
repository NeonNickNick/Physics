\documentclass[12pt, a4paper, oneside]{ctexart}
\usepackage{amsmath, amsthm, amssymb, graphicx}
\usepackage[bookmarks=true, colorlinks, citecolor=blue, linkcolor=black]{hyperref}
\usepackage{fancyhdr}
\usepackage{abstract}
\usepackage{xcolor}
\usepackage{geometry}
\geometry{a4paper,left=0.6in,right=0.6in,top=1.2in,bottom=1.2in}
\setlength\columnsep{0.5in}
\columnseprule=1pt
\pagestyle{fancy}
\fancyhf{}
\fancyhead[R]{\thepage} 
\fancyhead[C]{笔记} 
% 导言区
\title{\Huge 笔记}
\date{\today}
\setlength{\columnsep}{1in}
\newcommand{\blankspace}{\rule{2cm}{0.15mm}}
\begin{document}
	\maketitle
	\section{波函数}
	薛定谔方程
	\begin{equation}
		\mathrm{i}\hbar\dfrac{\partial\Psi}{\partial t}=-\dfrac{\hbar^{2}}{2m}\boldsymbol{\nabla}^{2}\Psi+V\Psi
	\end{equation}
	\quad\quad 或者写为
	\begin{equation}
		\mathrm{i}\hbar\dfrac{\partial\Psi}{\partial t}=\hat{H}\Psi
	\end{equation}
	\quad\quad 波恩的统计诠释认为波函数的模方表示发现粒子位于空间中某处的概率密度,即在$t$时刻在区域$\Omega$发现粒子的概率是
	\begin{equation}
		\iiint_{\Omega}\mathrm{d}^{3}x\left|\Psi\left(\boldsymbol{r},t\right)\right|^{2}
	\end{equation}
	\quad\quad 这种统计诠释引入了一种不确定性,即使已经知道了一个粒子的波函数,仍无法通过一次简单的位置测量来验证这个结果,因此波函数只能提供可能结果的统计信息。这种不确定性引发了一个问题:如果测量发现粒子在$C$处,那么在测量开始的前一刻粒子在哪里?有三种看法:\par 
	1.哥本哈根学派的观点:是测量导致粒子“位于$C$”处,换言之,观测本身不可避免地扰动了被观测量,导致波函数坍缩、观测到对应结果。这是被最广泛接受的观点,但它存在一个明显的问题:如果测量过程完全是一个真实存在的物理过程,它应该满足薛定谔方程,不应引起波函数坍缩;如果不是,那么和真实物理过程的界限在哪里?\par 
	2.现实主义学派的观点:粒子就在$C$点。这是爱因斯坦等人的观点。他们认为意味量子力学是不完备的,因为它没有办法告诉我们这一点。粒子的位置不是不可确定的,这说明除了$\Psi$之外还需要提供附加信息,称为隐变量。这一观点会导致更加令人不安的问题,考虑EPRB佯谬(EPR佯谬的一个简化版本):一个中性$\pi^{0}$介子的衰变过程
	\begin{equation}
		\pi^{0}\to\mathrm{e}^{-}+\mathrm{e}^{+}
	\end{equation}
	\quad\quad 由于$\pi^{0}$介子的自旋为$0$,由角动量守恒,正负电子的自旋必须相反,因此只需要测量电子的自旋,就立刻能够得知正电子自旋的测量结果,无论两个电子相距多远。假定任何作用的传播速度不能超过真空光速(定域性原则)。根据哥本哈根诠释,正负电子自旋的“产生”都是瞬间发生的,这明显存在矛盾;而现实主义学派的观点则避免了这个问题,因为二者的自旋从一开始就是确定的。\par 
	因此爱因斯坦等人声称为了完整描述物理实际需要额外引入隐变量$\lambda$。现在设置两个方向任意的探测器,考虑自旋乘积的平均值,假设两个探测器夹角为$\theta$,根据量子力学测量结果应该为
	\begin{equation}
		P(\boldsymbol{e_{1}},\boldsymbol{e_{2}})=-\dfrac{\hbar^{2}}{4}\cos\theta
	\end{equation}
	\quad\quad 根据现实主义学派的观点,如果用函数$A(\boldsymbol{e_{1}},\lambda)$、$B(\boldsymbol{e_{2}},\lambda)$表示正负电子的测量结果,这两个结果都只能取$\pm \dfrac{\hbar}{2}$,并且要满足角动量守恒
	\begin{equation}
		A(\boldsymbol{e},\lambda)+B(\boldsymbol{e},\lambda)=0
	\end{equation}
	\quad\quad 给隐变量引入一个概率密度,那么
	\begin{align}
		P(\boldsymbol{e_{1}},\boldsymbol{e_{2}})&=\int\mathrm{d}\lambda\rho(\lambda)A(\boldsymbol{e_{1}},\lambda)B(\boldsymbol{e_{2}},\lambda)\\
		&=-\int\mathrm{d}\lambda\rho(\lambda)A(\boldsymbol{e_{1}},\lambda)A(\boldsymbol{e_{2}},\lambda)\\
	\end{align}
	\quad\quad 这导致
	\begin{align}
		\left|P(\boldsymbol{e_{1}},\boldsymbol{e_{2}})-P(\boldsymbol{e_{1}},\boldsymbol{e_{3}})\right|&=\left|\int\mathrm{d}\lambda\rho(\lambda)\left[A(\boldsymbol{e_{1}},\lambda)A(\boldsymbol{e_{2}},\lambda)-A(\boldsymbol{e_{1}},\lambda)A(\boldsymbol{e_{3}},\lambda)\right]\right|\\
		&=\left|\int\mathrm{d}\lambda\rho(\lambda)\left[1-\dfrac{4}{\hbar^{2}}A(\boldsymbol{e_{2}},\lambda)A(\boldsymbol{e_{3}},\lambda)\right]A(\boldsymbol{e_{1}},\lambda)A(\boldsymbol{e_{2}},\lambda)\right|\\
		&\le\int\mathrm{d}\lambda\rho(\lambda)\left[\dfrac{\hbar^{2}}{4}-A(\boldsymbol{e_{2}},\lambda)A(\boldsymbol{e_{3}},\lambda)\right]\\
		&=\dfrac{\hbar^{2}}{4}+P(\boldsymbol{e_{2}},\boldsymbol{e_{3}})
	\end{align}
	\quad\quad 上式称为贝尔不等式,容易验证它与量子力学的结果是不相容的,而实验结果也证明贝尔不等式不成立,这说明定域性隐变量理论是错误的。\par 
	3.这个问题没有意义,因为测量前的状态不可能通过测量得到,换言之,得到的结果一定不是“测量前的”。\par 
	总而言之,这是量子力学中一个敏感的问题,至今没有得到非常圆满的解释。在之后将采用大多数人接受的哥本哈根学派的观点。\par 
	波函数必须满足归一化关系
	\begin{equation}
		\iiint\mathrm{d}^{3}x\left|\Psi(\boldsymbol{x},t)\right|^{2}=1
	\end{equation}
	\quad\quad 为了验证它总是归一化的,进行如下计算
	\begin{align}
		\dfrac{\mathrm{d}}{\mathrm{d}t}\iiint\mathrm{d}^{3}x\left|\Psi(\boldsymbol{x},t)\right|^{2}&=\iiint\mathrm{d}^{3}x\dfrac{\partial}{\partial t}\left|\Psi(\boldsymbol{x},t)\right|^{2}\\
		&=\mathrm{i}\dfrac{\hbar}{2m}\iiint\mathrm{d}^{3}x\left(\Psi^{*}\boldsymbol{\nabla}^{2}\Psi-\Psi\boldsymbol{\nabla}^{2}\Psi^{*}\right)\\
		&=0
	\end{align}
	\quad\quad 因此只要在某个时刻归一化了波函数,以后它都是归一化的。现在考虑动量
	\begin{equation}
		\left<\boldsymbol{p}\right>=m\dfrac{\mathrm{d}\left<x\right>}{\mathrm{d}t}=\iiint\mathrm{d}^{3}x\Psi^{*}\left(-\mathrm{i}\hbar\boldsymbol{\nabla}\right)\Psi
	\end{equation}
	\quad\quad 也可以写成(其实不应该混淆态和波函数,波函数实际上是态矢量在坐标表象下的投影。但是,如果始终默认选取坐标表象的话,是否加以区分影响不大)
	\begin{equation}
		\left<\boldsymbol{p}\right>=\left<\Psi|\boldsymbol{\hat{p}}\Psi\right>
	\end{equation}
	\quad\quad 此外还可以导出
	\begin{equation}
		\dfrac{\mathrm{d}\left<\boldsymbol{p}\right>}{\mathrm{d}t}=\left<-\boldsymbol{\nabla}V\right>
	\end{equation}
	\quad\quad 对于可观测量$\boldsymbol{Q}$,它要求
	\begin{equation}
		\left<\Psi\Big{|}\boldsymbol{\hat{Q}}\Psi\right>=\left<\boldsymbol{\hat{Q}}\Psi\big{|}\Psi\right>
	\end{equation}
	\quad\quad 这种算符与它的厄米共轭相等,称为厄米算符。根据测量公设,对于定态,测量结果将是其本征值$q$,此外还需要假设厄米算符的本征函数系是完备的。现在可以引入广义统计诠释:(为方便起见假设是离散谱)系统波函数可用本征函数展开
	\begin{equation}
		\Psi=c_{n}f_{n}
	\end{equation}
	\quad\quad 其中
	\begin{equation}
		c_{n}=\left<f_{n}|\Psi\right>
	\end{equation}
	\quad\quad 在一次测量中得到本征值$q_{k}$的概率是$|c_{n}|^{2}$,测量之后波函数坍缩至相应本征态。\par 
	现在还需要对易关系。它的灵感来自于泊松括号。定义
	\begin{equation}
		\left[\hat{A},\hat{B}\right]=\hat{A}\hat{B}-\hat{B}\hat{A}
	\end{equation}
	\quad\quad 可以计算一些典型的对易关系,例如
	\begin{align}
		\left[f(x),\hat{p}\right]&=\mathrm{i}\hbar\dfrac{\mathrm{d}f}{\mathrm{d}x}\\
		\left[r_{\alpha},\hat{p_{\beta}}\right]&=\mathrm{i}\hbar\delta_{\alpha\beta}\\
		\left[\hat{L_{\alpha}},\hat{L_{\beta}}\right]&=\mathrm{i}\hbar \epsilon_{\alpha\beta\gamma}\hat{L_{\gamma}}\\
		\left[\hat{L_{\alpha}},x_{\beta}\right]&=\mathrm{i}\hbar \epsilon_{\alpha\beta\gamma}x_{\gamma}\\
		\left[\hat{L_{\alpha}},\hat{p_{\beta}}\right]&=\mathrm{i}\hbar \epsilon_{\alpha\beta\gamma}\hat{p_{\gamma}}\\
	\end{align}
	\quad\quad 接下来就可以引入不确定原理。考虑两个可观测量$A$,$B$,计算测量二阶矩
	\begin{align}
		\hat{\sigma_{A}}&=\hat{A}-\left<A\right>\\
		\hat{\sigma_{B}}&=\hat{B}-\left<B\right>
	\end{align}
	\quad\quad 由施瓦茨不等式
	\begin{align}
		\sigma_{A}\sigma_{B}&=\sqrt{\left<\Psi\right|\hat{\sigma_{A}}^{2}\left|\Psi\right>\left<\Psi\right|\hat{\sigma_{B}}^{2}\left|\Psi\right>}\\
		&=\left|\left<\hat{\sigma_{A}}\Psi|\hat{\sigma_{B}}\Psi\right>\right|\\
		&\ge\dfrac{1}{2}\left|\left<\hat{\sigma_{A}}\Psi|\hat{\sigma_{B}}\Psi\right>-\left<\hat{\sigma_{B}}\Psi|\hat{\sigma_{A}}\Psi\right>\right|\\
		&=\dfrac{1}{2}\left|\left<\left[\hat{A},\hat{B}\right]\right>\right|
	\end{align}
	\quad\quad 例如典型的
	\begin{align}
		\sigma_{x_{\alpha}}\sigma_{p_{\beta}}=\dfrac{\hbar}{2}\delta_{\alpha\beta}
	\end{align}
	\quad\quad 我们早已知道如果两个可对角化矩阵是可交换的,那么总可以适当选择一个矩阵的本征矢量组使其全部都是另一个矩阵的本征矢量。这就说明可交换算符能构造共同本征态而不可交换算符一定不能。\par 
	对于可观测量$Q$,容易证明
	\begin{equation}
		\dfrac{\mathrm{d}\left<Q\right>}{\mathrm{d}t}=\left<\dfrac{\partial\hat{Q}}{\partial t}\right>+\dfrac{\mathrm{i}}{\hbar}\left<\left[\hat{H},\hat{Q}\right]\right>
	\end{equation}
	\quad\quad 假设$\hat{Q}$不含时
	\begin{equation}
		\sigma_{H}\sigma_{Q}\ge\dfrac{\hbar}{2}\left|\dfrac{\mathrm{d}\left<Q\right>}{\mathrm{d}t}\right|
	\end{equation}
	\quad\quad 得到能量-时间不确定关系
	\begin{align}
		&\Delta E=\sigma_{H}\\
		&\Delta t=\dfrac{\sigma_{Q}}{\left|\dfrac{\mathrm{d}\left<Q\right>}{\mathrm{d}t}\right|}\\
		&\Delta E\Delta t\ge\dfrac{\hbar}{2}
	\end{align}
	\section{定态薛定谔方程}
	如果势能不含时,可以将薛定谔方程分离变量得到含时摇摆因子
	\begin{equation}
		\mathrm{e}^{-\mathrm{i}\frac{E}{\hbar}t}
	\end{equation}
	\quad\quad 与定态薛定谔方程
	\begin{equation}
		\hat{H}\psi=E\psi
	\end{equation}
	\quad\quad 对于谐振子模型,构造升降算符(两个升降算符互为厄米共轭算符)
	\begin{align}
		\hat{a_{\pm}}&=\dfrac{1}{\sqrt{2m\hbar\omega}}\left(m\omega x\mp\mathrm{i}\hat{p}\right)\\
		\hat{H}&=\hbar\omega\left(\hat{a_{\mp}}\hat{a_{\pm}}\mp\dfrac{1}{2}\right)
	\end{align}
	\quad\quad 假设现在已经找到了本征态$\psi$,容易推出
	\begin{align}
		\hat{H}\left(\hat{a_{\pm}\psi}\right)&=\hbar\omega\left(\hat{a_{\pm}}\hat{a_{\mp}}\hat{a_{\pm}}\mp\dfrac{1}{2}\hat{a_{\pm}}\right)\psi\\
		&=\hat{a_{\pm}}\left(\hat{H}\pm\hbar\omega\right)\psi\\
		&=\left(E\pm\hbar\omega\right)\hat{a_{\pm}}\psi
	\end{align}
	\quad\quad 由于能量不可能一直降低,一定存在基态满足
	\begin{align}
		\hat{a_{-}}\psi_{0}=0\to\dfrac{\mathrm{d}\psi_{0}}{\mathrm{d}x}+\dfrac{m\omega}{\hbar}x\psi_{0}=0
	\end{align}
	\quad\quad 可以看出解为
	\begin{align}
		\psi_{0}&=\left(\dfrac{m\omega}{\pi\hbar}\right)^{\frac{1}{4}}\mathrm{e}^{-\frac{m\omega}{2\hbar}x^{2}}\\
		E_{0}&=\dfrac{1}{2}\hbar\omega
	\end{align}
	\quad\quad 为了定出各个激发态的归一化系数,可以考虑计算
	\begin{align}
		\hat{a_{+}}\hat{a_{-}}\psi_{n}&=n\psi_{n}\\
		\hat{a_{-}}\hat{a_{+}}\psi_{n}&=(n+1)\psi_{n}\\
	\end{align}
	\quad\quad 与$\psi_{n}$取内积得到
	\begin{align}
		\hat{a_{-}}\psi_{n}&=\sqrt{n}\psi_{n}\\
		\hat{a_{+}}\psi_{n}&=\sqrt{n+1}\psi_{n}\\
	\end{align}
	\quad\quad 因此
	\begin{equation}
		\psi_{n}=\dfrac{1}{\sqrt{n}}\left(\hat{a_{+}}\right)^{n}\psi_{0}
	\end{equation}
	\quad\quad 现在转而讨论氢原子。为了简单起见,先用角动量算符表示拉普拉斯算符
	\begin{equation}
		\boldsymbol{\nabla}^{2}=\dfrac{1}{r^{2}}\dfrac{\partial}{\partial r}\left(r^{2}\dfrac{\partial}{\partial r}\right)-\dfrac{\boldsymbol{\hat{L}}^{2}}{\hbar r^{2}}
	\end{equation}
	\quad\quad 我们早已知道分离变量解的角向部分是球谐函数,满足
	\begin{equation}
		\dfrac{\boldsymbol{\hat{L}}^{2}}{\hbar}\mathrm{Y}_{l}^{m}(\theta,\phi)=l(l+1)\mathrm{Y}_{l}^{m}(\theta,\phi)
	\end{equation}
	\quad\quad 这样很快得到了径向方程
	\begin{equation}
		-\dfrac{\hbar^{2}}{2m}\left[\dfrac{1}{r}\dfrac{\mathrm{d}}{\mathrm{d}r}\left(r^{2}\dfrac{\mathrm{d}R}{\mathrm{d}r}\right)-\dfrac{l\left(l+1\right)R}{r}\right]-\dfrac{e^{2}R}{4\pi\varepsilon_{0}}=ERr
	\end{equation}
	\quad\quad 其实这里我们可以从光学中球面波获得启发,作换元$u=Rr$,$\kappa=\dfrac{\sqrt{-2mE}}{\hbar}$这样得到
	\begin{equation}
		\dfrac{1}{\kappa^{2}}\dfrac{\mathrm{d}^{2}u}{\mathrm{d}r^{2}}=\left[1-\dfrac{me^{2}}{2\pi\varepsilon_{0}\hbar^{2}\kappa}\dfrac{1}{\kappa r}+\dfrac{l\left(l+1\right)}{\left(\kappa r\right)^{2}}\right]u
	\end{equation}
	\quad\quad 进一步令$\rho=\kappa r$,$\rho_{0}=\dfrac{me^{2}}{2\pi\varepsilon_{0}\hbar^{2}\kappa}$得到
	\begin{equation}
		\dfrac{\mathrm{d}^{2}u}{\mathrm{d}\rho^{2}}=\left[1-\dfrac{\rho_{0}}{\rho}+\dfrac{l\left(l+1\right)}{\rho^{2}}\right]u
	\end{equation}
	\quad\quad 渐进行为要求我们将解的形式设为(这是因为我们不希望讨论洛朗级数)
	\begin{equation}
		u=\rho^{l+1}\mathrm{e}^{-\rho}v
	\end{equation}
	\quad\quad 即
	\begin{equation}
		\rho\dfrac{\mathrm{d}^{2}v}{\mathrm{d}\rho^{2}}+2(l+1-\rho)\dfrac{\mathrm{d}v}{\mathrm{d}\rho}+\left[\rho_{0}-2(l+1)\right]v=0
	\end{equation}
	\quad\quad 现在可以带入幂级数解
	\begin{equation}
		v=\sum_{j=0}^{+\infty}c_{j}\rho^{j}
	\end{equation}
	\quad\quad 整理得
	\begin{equation}
		\dfrac{c_{j+1}}{c_{j}}=\dfrac{2\left(j+l+1\right)-\rho_{0}}{\left(j+1\right)\left(j+2l+2\right)}
	\end{equation}
	\quad\quad 如果有无穷多项,当$j$充分大的时候$v\sim\mathrm{e}^{2\rho}$,因此必须截断,即
	\begin{align}
		\rho_{0}&=2n\\
		N&=n-l\ge1\\
		c_{N}&=0
	\end{align}
	\quad\quad 这样得到了能级公式
	\begin{equation}
		E_{n}=-\dfrac{m}{2\hbar^{2}}\left(\dfrac{e^{2}}{4\pi\varepsilon_{0}}\right)^{2}\dfrac{1}{n^{2}}
	\end{equation}
	\quad\quad 定义玻尔半径
	\begin{align}
		a&=\dfrac{4\pi\varepsilon_{0}\hbar^{2}}{me^{2}}\\
		\rho&=\dfrac{r}{an}
	\end{align}
	\quad\quad 现在将波函数组合为
	\begin{equation}
		\psi_{nlm}=A\mathrm{e}^{-r / n a}\left(\frac{2 r}{n a}\right)^{l}\left[\mathrm{L}_{n-l-1}^{2 l+1}(2 r / n a)\right] \mathrm{Y}_{l}^{m}(\theta, \phi)
	\end{equation}
	\quad\quad 其中缔合拉盖尔多项式
	\begin{equation}
	\mathrm{L}^{p}_{q}(x)=\dfrac{(-1)^{p}}{(p+q)!}\dfrac{\mathrm{d}^{p}}{\mathrm{d}x^{p}}\left[\mathrm{e}^{x}\dfrac{\mathrm{d}^{p+q}}{\mathrm{d}x^{p+q}}\left(e^{-x}x^{p+q}\right)\right]
	\end{equation}
	\quad\quad 计算归一化系数\footnote{其中最有技巧性地一步是凑出二项式展开。坦白讲这一步有从答案凑过程的嫌疑,但是我认为交换分母阶乘因子构造新的组合数是一种常用的技巧,这一步还是较为直观的,至少没有数学知识和记号上的困难。}
	\begin{align}
		&\iiint\mathrm{d}^{3}x\left|\psi_{nlm}\right|^{2}\\&=A^{2}\left(\dfrac{na}{2}\right)^{3}\sum_{p=0}^{n-l-1}\sum_{q=0}^{n-l-1}\int_{0}^{+\infty}\dfrac{\left(-1\right)^{p+q}\left[\left(n+l\right)!\right]^{2}x^{2n-p-q}\mathrm{e}^{-x}}{p!q!\left(n+l-p\right)!\left(n+l-q\right)!\left(n-l-1-p\right)!\left(n-l-1-q\right)!}\mathrm{d}x\\
		&=A^{2}\left(\dfrac{na}{2}\right)^{3}\left[\left(n+l\right)!\right]^{2}\sum_{p=0}^{n-l-1}\sum_{q=0}^{n-l-1}\dfrac{\left(-1\right)^{p+q}\left(2n-p-q\right)!}{p!q!\left(n+l-p\right)!\left(n+l-q\right)!\left(n-l-1-p\right)!\left(n-l-1-q\right)!}\\
		&=A^{2}\left(\dfrac{na}{2}\right)^{3}\dfrac{\left[\left(n+l\right)!\right]^{2}}{\left(n-l-1\right)!}\sum_{p=0}^{n-l-1}\dfrac{\left(-1\right)^{n+l+p}\dfrac{\mathrm{d}^{n-l-p}}{\mathrm{d}x^{n-l-p}}\left[x^{n+l+1-p}\left(1+x\right)^{n-l-1}\right]\Big{|}_{x=-1}}{p!\left(n-l-1-p\right)!\left(n+l-p\right)!}\\		&=A^{2}\left(\dfrac{na}{2}\right)^{3}\dfrac{\left[\left(n+l\right)!\right]^{2}}{\left(n-l-1\right)!}\sum_{p=0}^{\min\left\{1,n-l-1\right\}}\dfrac{\left(-1\right)^{n+l+p}\dfrac{\mathrm{d}^{n-l-p}}{\mathrm{d}x^{n-l-p}}\left[x^{n+l+1-p}\left(1+x\right)^{n-l-1}\right]\Big{|}_{x=-1}}{p!\left(n-l-1-p\right)!\left(n+l-p\right)!}\\
		&=A^{2}\left(\dfrac{na}{2}\right)^{3}\dfrac{2n\left(n+l\right)!}{\left(n-l-1\right)!}\\
		&=1
	\end{align}
	\quad\quad 经过冗长的计算得到定态波函数
	\begin{equation}
		\psi_{nlm}=\sqrt{\left(\frac{2}{n a}\right)^{3} \frac{(n-l-1)!}{2 n(n+l)!}} \mathrm{e}^{-r / n a}\left(\frac{2 r}{n a}\right)^{l}\left[\mathrm{L}_{n-l-1}^{2 l+1}(2 r / n a)\right] \mathrm{Y}_{l}^{m}(\theta, \phi)
	\end{equation}
	\quad\quad 接下来再看几个例子。\par 
	考虑哈密顿量
	\begin{equation}
		\hat{H}(\lambda)=\dfrac{\hat{p}^{2}}{2m}+V+\lambda\hat{p}
	\end{equation}
	\quad\quad 目前已经知道$\hat{H}(0)$的本征函数和对应能级
	\begin{equation}
		\psi_{n},E_{n}
	\end{equation}
	\quad\quad 现在来求$\hat{H}(\lambda)$的本征函数和对应能级。我们先把哈密顿量写成
	\begin{equation}
		\hat{H}=\dfrac{\left(\hat{p}+m\lambda\right)^{2}}{2m}+V-\dfrac{1}{2}m\lambda^{2}
	\end{equation}
	\quad\quad 这告诉我们这个问题在动量表象下比较简单,添加的常数定价于坐标平移,这样可以直接写出能级
	\begin{equation}
		E_{n}(\lambda)=E_{n}-\dfrac{1}{2}m\lambda^{2}
	\end{equation}
	\quad\quad 以及波函数
	\begin{align}
		\psi_{n}(x,\lambda)&=\dfrac{1}{\sqrt{2\pi\hbar}}\int\mathrm{d}p\mathrm{e}^{\mathrm{i}\frac{px}{\hbar}}\dfrac{1}{\sqrt{2\pi\hbar}}\int\mathrm{d}x^{\prime}\mathrm{e}^{-\mathrm{i}\frac{\left(p+m\lambda\right)x^{\prime}}{\hbar}}\psi_{n}\left(x^{\prime}\right)\\
		&=\mathrm{e}^{-\mathrm{i}\frac{m\lambda}{\hbar}x}\psi(x)
	\end{align}
	\quad\quad 考虑哈密顿量
	\begin{equation}
		\overleftrightarrow{H}=\begin{pmatrix}
		\dfrac{5}{2}\hat{a_{-}}\hat{a_{+}}+\hat{a_{-}}^{2}+\hat{a_{+}}^{2}+\hat{a_{-}}+\hat{a_{+}}	& \dfrac{1}{5}\\ 
		\dfrac{1}{5}	&\dfrac{5}{2}\hat{a_{-}}\hat{a_{+}}+\hat{a_{-}}^{2}+\hat{a_{+}}^{2}+\hat{a_{-}}+\hat{a_{+}}
		\end{pmatrix}
	\end{equation}
	\quad\quad 我们希望求它的特征值。这种问题可以重新构造升降算符(系数均为实数)
	\begin{align}
		\hat{b_{-}}&=\alpha\hat{a_{-}}+\beta\hat{a_{+}}+\gamma\\
		\hat{b_{+}}&=\alpha\hat{a_{+}}+\beta\hat{a_{-}}+\gamma
	\end{align}
	\quad\quad 升降算符要求
	\begin{align}
		\hat{a_{+}}\hat{a_{-}}\psi_{n}&=n\psi_{n}\\
		\hat{a_{-}}\hat{a_{+}}\psi_{n}&=\left(n+1\right)\psi_{n}
	\end{align}
	\quad\quad 我们期望维持对易关系不变,且凑出哈密顿量中的系数,这要求
	\begin{equation}
		\left\{\begin{matrix}
		\alpha^{2}-\beta^{2}=1	\\
		\alpha^{2}+\beta^{2}=\dfrac{5}{2}\alpha\beta	\\
		\gamma\left(\alpha+\beta\right)=\alpha\beta
		\end{matrix}\right.\to
		\left\{\begin{matrix}
			\alpha=\dfrac{2\sqrt{3}}{3} \\
			\beta=\dfrac{\sqrt{3}}{3}\\
			\alpha\beta=\dfrac{2}{3}\\
			\gamma=\dfrac{2\sqrt{3}}{9}
		\end{matrix}\right.
	\end{equation}
	\quad\quad 因此
	\begin{equation}
		\hat{A}=\dfrac{5}{2}\hat{a_{-}}\hat{a_{+}}+\hat{a_{-}}^{2}+\hat{a_{+}}^{2}+\hat{a_{-}}+\hat{a_{+}}=\dfrac{3}{2}\hat{b_{+}}\hat{b_{-}}+\dfrac{16}{9}
	\end{equation}
	\quad\quad 我们按照和以前完全相同的方法
	\begin{align}
		\hat{A}\left|n\right>&=\lambda\left|n\right>\\
		\hat{A}\left(\hat{b_{\pm}}\left|n\right>\right)&=\left(\lambda\pm\dfrac{3}{2}\right)\left|n\right>\\
		\hat{b_{-}}\left|0\right>&=0
	\end{align}
	\quad\quad 因此得到了一组本征态与对应能级
	\begin{equation}
		\left|n\right>,\lambda_{n}=\dfrac{3}{2}n+\dfrac{16}{9},n=0,1,2\dots
	\end{equation}
	\quad\quad 对于原哈密顿量,比较直观地构造一组基
	\begin{equation}
		\begin{pmatrix}
		\left|0\right>	\\
			0
		\end{pmatrix},
		\begin{pmatrix}
			\left|1\right>	\\
			0
		\end{pmatrix},
		\begin{pmatrix}
			\left|2\right>	\\
			0
		\end{pmatrix}\dots
		\begin{pmatrix}
				0	\\
		\left|0\right>
		\end{pmatrix},
		\begin{pmatrix}
			0	\\
			\left|1\right>
		\end{pmatrix},
		\begin{pmatrix}
			0	\\
			\left|2\right>
		\end{pmatrix}\dots
	\end{equation}
	\quad\quad 哈密顿量矩阵在这个基下被化为四个对角矩阵,利用
	\begin{equation}
		\begin{vmatrix}
		\overleftrightarrow{A}	&\overleftrightarrow{B} \\
		\overleftrightarrow{B}	&\overleftrightarrow{A}
		\end{vmatrix}=\left|\overleftrightarrow{A}+\overleftrightarrow{B}\right|\left|\overleftrightarrow{A}-\overleftrightarrow{B}\right|
	\end{equation}
	\quad\quad 得到特征值
	\begin{align}
		\left\{\begin{matrix}
		\lambda_{n1}&=\dfrac{3}{2}n+\dfrac{89}{45}	\\\\
		\lambda_{n1}&=\dfrac{3}{2}n+\dfrac{71}{45}
		\end{matrix}\right.,n=0,1,2\dots
	\end{align}
	\quad\quad 其实这个例子还告诉我们可以用本征态表示升降算符
	\begin{equation}
		y
	\end{equation}
	\section{自旋}
	接下来将要讨论一个略显微妙的问题。容易计算对易关系
	\begin{equation}
		\left[\boldsymbol{\hat{L}}^{2},\hat{L_{z}}\right]=0
	\end{equation}
	\quad\quad 我们期待能够找到共同本征态
	\begin{align}
		\boldsymbol{\hat{L}}^{2}f&=\lambda f\\
		\hat{L_{z}}f&=\mu f
	\end{align}
	\quad\quad 仍然构造升降算符
	\begin{align}
		\hat{L_{\pm}}&=\hat{L_{x}}\pm\mathrm{i}\hat{L_{y}}\\
		\boldsymbol{\hat{L}}^{2}\hat{L_{\pm}}f&=\hat{L_{\pm}}\boldsymbol{\hat{L}}^{2}f=\lambda\boldsymbol{\hat{L}}^{2}\hat{L_{\pm}}f\\
		\hat{L_{z}}\hat{L_{\pm}}f&=\hat{L_{\pm}}\hat{L_{z}}f\pm\hbar\hat{L_{\pm}}f=(\lambda\pm\hbar)\hat{L_{\pm}}f
	\end{align}
	\quad\quad 由于分量的平方不可能超过总量的平方,必然有上下限
	\begin{align}
		\hat{L_{+}}f_{top}&=0\\
		\hat{L_{z}}f_{top}&=l\hbar\\
		\boldsymbol{\hat{L}}^{2}f_{top}&=\lambda\\
		\hat{L_{-}}f_{bottom}&=0\\
		\hat{L_{z}}f_{bottom}&=\overline{l}\hbar\\
		\boldsymbol{\hat{L}}^{2}f_{bottom}&=\lambda\\
	\end{align}
	\quad\quad 注意到有如下展开
	\begin{equation}
		\boldsymbol{\hat{L}}^{2}=\hat{L_{\pm}}\hat{L_{\mp}}\mp\hbar\hat{L_{z}}+\hat{L_{z}}^{2}
	\end{equation}
	\quad\quad 因此
	\begin{align}
		\lambda=\hbar^{2}l(l+1)&=\hbar^{2}\overline{l}(\overline{l}-1)\\
		\overline{l}&=-l
	\end{align}
	\quad\quad 现在我们发现最后的形式是我们熟悉的
	\begin{align}
		\boldsymbol{\hat{L}}^{2}f^{m}_{l}&=\hbar^{2}l(l+1) f^{m}_{l}\\
		\hat{L_{z}}f^{m}_{l}&=\hbar mf^{m}_{l}
	\end{align}
	\quad\quad 通过与之前类似的方法可以得到
	\begin{align}
		\hat{L_{+}}f^{m}_{l}&=\sqrt{l(l+1)-m(m+1)}f^{m+1}_{l}\\
		\hat{L_{-}}f^{m}_{l}&=\sqrt{l(l+1)-m(m-1)}f^{m-1}_{l}
	\end{align}
	\quad\quad 我们发现这里的角动量$z$分量本征值并不限定为$\hbar$的整数倍,还可以是半整数倍。这是因为这里讨论的是一般形式的角动量。之前在坐标表象下求解角动量本征值时受到波函数单值性的限制,本征值只能取整数,而接下来要讨论的自旋角动量可以取半整数。在非相对论量子力学中只能认为自旋是粒子的内禀属性,除此之外不做任何深入探讨,自然也不能像之前那样写出波函数,但仍可用态矢量表示。现在先把它当作轨道角动量理论的翻版构造对易关系
	\begin{equation}
		\left[\hat{S_{\alpha}},\hat{S_{\beta}}\right]=\mathrm{i}\hbar \epsilon_{\alpha\beta\gamma}\hat{S_{\gamma}}
	\end{equation}
	\quad\quad 同样地
	\begin{align}
		\boldsymbol{\hat{S}}^{2}\left|s,m\right>&=\hbar^{2}l(l+1)\left|s,m\right>\\
		\hat{S_{z}}\left|s,m\right>&=\hbar m\left|s,m\right>\\
		\hat{S_{+}}\left|s,m\right>&=\sqrt{s(s+1)-m(m+1)}\left|s,m+1\right>\\
		\hat{S_{-}}\left|s,m\right>&=\sqrt{s(s+1)-m(m-1)}\left|s,m-1\right>
	\end{align}
	\quad\quad 最简单的情况是$s=\dfrac{1}{2}$,共有两个本征态
	\begin{equation}
		\left|\dfrac{1}{2},\dfrac{1}{2}\right>,\left|\dfrac{1}{2},-\dfrac{1}{2}\right>
	\end{equation}
	\quad\quad 这种粒子的状态可以用旋量表示
	\begin{equation}
		\chi=\begin{pmatrix}
			a\\b
		\end{pmatrix}
		=a\chi_{+}+b\chi_{-}=a\begin{pmatrix}
			1\\0
		\end{pmatrix}+b\begin{pmatrix}
		0\\1
		\end{pmatrix}
	\end{equation}
	\quad\quad 这里其实也不应该混淆态矢量和旋量。态矢量属于希尔伯特空间,而旋量是相对特定基矢量的一组分量。目前这种情况可以用矩阵表示自旋算符(在没有歧义的情况下不区分张量、矩阵和算符,仍沿用比较直观的符号)
	\begin{align}
		\hat{S_{z}}&=\dfrac{\hbar}{2}\overleftrightarrow{\sigma_{z}}=\dfrac{\hbar}{2}\begin{pmatrix}
		1	& 0\\
		0	&-1
		\end{pmatrix}\\
		\hat{S_{x}}&=\dfrac{\hbar}{2}\overleftrightarrow{\sigma_{x}}=\dfrac{\hbar}{2}\begin{pmatrix}
			0	& 1\\
			1	& 0
		\end{pmatrix}\\
		\hat{S_{y}}&=\dfrac{\hbar}{2}\overleftrightarrow{\sigma_{y}}=\dfrac{\hbar}{2}\begin{pmatrix}
		0	& -\mathrm{i}\\
		\mathrm{i}	& 0
		\end{pmatrix}\\
		\boldsymbol{\hat{S}}^{2}&=\dfrac{3\hbar^{2}}{4}\begin{pmatrix}
			1	& 0\\
			0	&1
		\end{pmatrix}
	\end{align}
	\quad\quad 容易验证对易关系
	\begin{equation}
		\overleftrightarrow{\sigma_{\alpha}}\overleftrightarrow{\sigma_{\beta}}=\delta_{\alpha\beta}\overleftrightarrow{I}+\mathrm{i}\epsilon_{\alpha\beta\gamma}\overleftrightarrow{\sigma_{\gamma}}
	\end{equation}
	\section{电磁作用}
	电磁作用的一个经典例子是拉莫尔进动。假设$s=\dfrac{1}{2}$粒子静止在磁场$\boldsymbol{B}=B_{0}\boldsymbol{e_{z}}$中,哈密顿量为
	\begin{equation}
		\overleftrightarrow{H}=-\gamma B_{0}\overleftrightarrow{S_{z}}=-\dfrac{\gamma B_{0}\hbar}{2}\begin{pmatrix}
			1	& 0\\
			0	&-1
		\end{pmatrix}
	\end{equation}
	\quad\quad 只需求解本征值问题
	\begin{equation}
		\overleftrightarrow{H}\chi=E\chi
	\end{equation}
	\quad\quad 很容易解出
	\begin{equation}
		\chi(t)=\begin{pmatrix}
		\cos(\frac{\alpha}{2})\mathrm{e}^{\frac{\mathrm{i}\gamma B_{0}}{2}t}	\\\sin(\frac{\alpha}{2})\mathrm{e}^{-\frac{\mathrm{i}\gamma B_{0}}{2}t}
		\end{pmatrix}
	\end{equation}
	\quad\quad 计算期望值
	\begin{align}
		\left<S_{x}\right>&=\chi^{\dagger}\overleftrightarrow{S_{x}}\chi=\dfrac{\hbar}{2}\sin\alpha\cos\left(\gamma B_{0}t\right)\\
		\left<S_{y}\right>&=\chi^{\dagger}\overleftrightarrow{S_{x}}\chi=-\dfrac{\hbar}{2}\sin\alpha\sin\left(\gamma B_{0}t\right)\\
		\left<S_{z}\right>&=\chi^{\dagger}\overleftrightarrow{S_{x}}\chi=\dfrac{\hbar}{2}\cos\alpha
	\end{align}
	\quad\quad 现在将外场改为$\boldsymbol{B}=B_{0}\cos\left(\omega t\right)\boldsymbol{e_{z}}$,哈密顿量变为
	\begin{equation}
		\overleftrightarrow{H}=-\gamma B_{0}\overleftrightarrow{S_{z}}=-\dfrac{\gamma B_{0}\hbar}{2}\cos\left(\omega t\right)\begin{pmatrix}
			1	& 0\\
			0	&-1
		\end{pmatrix}
	\end{equation}
	\quad\quad 这里必须求解薛定谔方程
	\begin{align}
		\mathrm{i}\dfrac{\mathrm{d}}{\mathrm{d}t}\chi&=-\dfrac{\gamma B_{0}}{2}\cos\left(\omega t\right)\begin{pmatrix}
			1	& 0\\
			0	&-1
		\end{pmatrix}\chi\\
		\chi(0)&=\dfrac{1}{\sqrt{2}}\begin{pmatrix}
			1\\
			1
		\end{pmatrix}=\chi_{+}^{\left(x\right)}
	\end{align}
	\quad\quad 容易解得
	\begin{equation}
		\chi(t)=\dfrac{1}{\sqrt{2}}\begin{pmatrix}
			\mathrm{e}^{-\mathrm{i}\frac{\gamma B{0}}{2\omega}\sin(\omega t)}\\
			\mathrm{e}^{\mathrm{i}\frac{\gamma B{0}}{2\omega}\sin(\omega t)}
		\end{pmatrix}=\cos\left[\dfrac{\gamma B_{0}}{2\omega}\sin\left(\omega t\right)\right]\chi_{+}^{\left(x\right)}-\mathrm{i}\sin\left[\dfrac{\gamma B_{0}}{2\omega}\sin\left(\omega t\right)\right]\chi_{-}^{\left(x\right)}
	\end{equation}
	\quad\quad 可以发现当$B_{0}\ge\pi\dfrac{\omega}{\gamma}$时,存在某些时刻自旋态完全翻转。\par
	另一个例子是A-B效应。首先写出电磁场中带电粒子的哈密顿量,由于需要满足对易关系,这里的动量是正则动量
	\begin{equation}
		\hat{H}=\dfrac{1}{2m}\left(\boldsymbol{\hat{p}}-q\boldsymbol{A}\right)^{2}+q\phi=\dfrac{1}{2m}\left(-\mathrm{i}\hbar\boldsymbol{\nabla}-q\boldsymbol{A}\right)^{2}+q\phi
	\end{equation}
	\quad\quad 容易得\footnote{最好是全部拆成指标计算,这样不容易出错}
	\begin{align}
		\dfrac{\mathrm{d}\left<\boldsymbol{\hat{p}}-q\boldsymbol{A}\right>}{\mathrm{d}t}&=\dfrac{\mathrm{i}}{\hbar}\left<\left[\hat{H},\boldsymbol{\hat{p}}-q\boldsymbol{A}\right]\right>\\
		&=q\left<\boldsymbol{E}\right>+\dfrac{\mathrm{i}}{2m\hbar}\left<\left[\left(\boldsymbol{\hat{p}}-q\boldsymbol{A}\right)^{2},\boldsymbol{\hat{p}}-q\boldsymbol{A}\right]\right>\\
		&=q\left<\boldsymbol{E}\right>+\dfrac{q}{2m}\left<\boldsymbol{\hat{p}}\times\boldsymbol{B}-\boldsymbol{B}\times\boldsymbol{\hat{p}}\right>-\dfrac{q^{2}}{m}\left<\boldsymbol{A}\times\boldsymbol{B}\right>
	\end{align}
	\quad\quad 定义机械动量
	\begin{equation}
		\boldsymbol{P_{mec}}=\boldsymbol{p}-q\boldsymbol{A}
	\end{equation}
	\quad\quad 结果等价于
	\begin{equation}
		\dfrac{\mathrm{d}\left<\boldsymbol{P_{mec}}\right>}{\mathrm{d}t}=q\left<\boldsymbol{E}\right>+\dfrac{q}{2m}\left<\boldsymbol{\hat{P_{mec}}}\times\boldsymbol{B}-\boldsymbol{B}\times\boldsymbol{\hat{P_{mec}}}\right>
	\end{equation}
	\quad\quad 如果电磁场是均匀的(尤其是磁场是均匀的)
	\begin{equation}
	\dfrac{\mathrm{d}\left<\boldsymbol{P_{mec}}\right>}{\mathrm{d}t}=q\left(\boldsymbol{E}+\dfrac{\left<\boldsymbol{P_{mec}}\right>}{m}\times\boldsymbol{B}\right)
	\end{equation}
	\quad\quad 可以发现形式上与经典情况相同。现在讨论AB效应。将无旋度区域波函数设为(这个波函数的适用路径集合中的任意两条路径构成的回路磁通量为$0$)
	\begin{equation}
		\Psi=\mathrm{e}^{\mathrm{i}\frac{q}{\hbar}\int^{\boldsymbol{r}}\boldsymbol{A}(\boldsymbol{r^{\prime}},t)\cdot\mathrm{d}\boldsymbol{r^{\prime}}}\Psi_{0}
	\end{equation}
	\quad\quad 带入薛定谔方程得
	\begin{equation}
		\mathrm{i}\hbar\dfrac{\partial\Psi_{0}}{\partial t}=-\dfrac{\hbar^{2}}{2m}\boldsymbol{\nabla}^{2}\Psi_{0}+q\Psi_{0}\int^{\boldsymbol{r}}\left(\boldsymbol{\nabla}\phi+\dfrac{\partial\boldsymbol{A}}{\partial t}\right)\cdot\mathrm{d}\boldsymbol{r^{\prime}}
	\end{equation}
	\quad\quad 这部分因子对相位无影响\footnote{事实上,这是一个值得商榷的问题。虽然我们限定这个波函数构造于$\boldsymbol{\nabla}\times\boldsymbol{A}=0$的区域,但这个条件不保证积分是单值的。只要两条路径之间的磁通量不为$0$,积分结果就不同,这导致$\Psi_{0}$其实不是单值的。同理,由于两条路径之间磁通量的变化率可能不为$0$,化简后薛定谔方程中包含的积分也不一定是单值的。所以我认为,在理解AB效应时(至少在教材范围内),应该认为粒子依然是近似经典的,它的波包局限在一个并不大的范围内,且$\boldsymbol{E}$是保守场(这样解决了$\Psi_{0}$的多值性问题,这往往要求我们考虑的问题是定态的),这时候将$\boldsymbol{A}$的作用理解为是一个附加的波矢,就能够理解之后的结果。在这里,可能还有另一个疑问:假如通过某种非电磁手段使得粒子绕有旋区域运动一周(由于认为粒子依然是经典定域的,可以谈论具体的运动路径),它的相位有什么变化?这实际上使我们不得不面对多值性的问题。无论我们怎么划分波函数的单值区间,粒子都会越过割线。对于这个问题,我仍然选择将矢势看作是一个附加的波矢,这样波函数就能连续过渡到另一个分支,至少对于我而言这是一个物理上比较容易接受的解释。},因此相位差为
	\begin{equation}
		\Delta\varphi=\dfrac{q}{\hbar}\oint_{C^{+}}\boldsymbol{A}(\boldsymbol{r^{\prime}},t)\cdot\mathrm{d}\boldsymbol{r^{\prime}}=\dfrac{q\Phi}{\hbar}
	\end{equation}
	\quad\quad 很容易看出来这一结果与规范无关,因为任意规范变换
	\begin{equation}
		\phi\to\phi-\dfrac{\partial\Lambda}{\partial t},\boldsymbol{A}\to\boldsymbol{A}+\boldsymbol{\nabla}\Lambda
	\end{equation}
	\quad\quad 不改变
	\begin{equation}
		\boldsymbol{\nabla}\phi+\dfrac{\partial\boldsymbol{A}}{\partial t}
	\end{equation}
	\quad\quad 接下来还可以再看几个例子。考虑电磁势
	\begin{align}
		\boldsymbol{A}&=\dfrac{B_{0}}{2}\left(x\boldsymbol{e_{y}}-y\boldsymbol{e_{x}}\right)\\
		\phi&=Kz^{2}
	\end{align}
	\quad\quad 定态薛定谔方程为
	\begin{equation}
		-\dfrac{\hbar^{2}}{2m}\boldsymbol{\nabla}^{2}\psi-\dfrac{q B_{0}}{2m}\hat{L_{z}}\psi+\dfrac{q^{2}B_{0}^{2}}{8m}\left(x^{2}+y^{2}\right)\psi+qKz^{2}\psi=E\psi
	\end{equation}
	\quad\quad 可以发现$\left[\hat{H},\hat{L_{z}}\right]=0$,从而可以找到共同本征态\footnote{实际上能够发现这个体系对应的经典情形在$\boldsymbol{e_{z}}$方向角动量守恒}(目前我们并不知道本征值是多少)
	\begin{equation}
		-\dfrac{\hbar^{2}}{2m}\boldsymbol{\nabla}^{2}\psi-\dfrac{q B_{0}\hbar}{2m}m_{z}\psi+\dfrac{q^{2}B_{0}^{2}}{8m}r^{2}\psi+qKz^{2}\psi=E\psi
	\end{equation}
	\quad\quad 换元并分离变量得
	\begin{equation}
		\left\{-\frac{\hbar^{2}}{2 m}\left[\frac{1}{R r} \frac{\mathrm{d}}{\mathrm{d} r}\left(r \frac{\mathrm{d} R}{\mathrm{d} r}\right)-\frac{m_{z}^{2}}{r^{2}}\right]+\frac{1}{8} m \omega_{1}^{2} r^{2}\right\}+\left(-\frac{\hbar^{2}}{2 m} \frac{1}{Z} \frac{\mathrm{d}^{2} Z}{\mathrm{d} z^{2}}+\frac{1}{2} m \omega_{2}^{2} z^{2}\right) \\
		=E+\frac{m_{z} \hbar \omega_{1}}{2}
	\end{equation}
	\quad\quad 很容易看出$z$方向是一个一维谐振子,因此
	\begin{equation}
		-\frac{\hbar^{2}}{2 m}\left[\frac{1}{R r} \frac{\mathrm{d}}{\mathrm{d} r}\left(r \frac{\mathrm{d} R}{\mathrm{d} r}\right)-\frac{m_{z}^{2}}{r^{2}}\right]+\frac{1}{8} m \omega_{1}^{2} r^{2}=E+\dfrac{m_{z}\hbar\omega_{1}}{2}-\left(n_{2}+\dfrac{1}{2}\right)\omega_{2}
	\end{equation}
	\quad\quad 令$u=\sqrt{r}R$
	\begin{equation}
		-\dfrac{\hbar^{2}}{2m}\dfrac{\mathrm{d}^{2}u}{\mathrm{d}r^{2}}+\left[\dfrac{1}{8}m\omega_{1}^{2}r^{2}+\dfrac{\hbar^{2}}{2m}\dfrac{\left(m_{z}^{2}-\dfrac{1}{4}\right)}{r^{2}}\right]u=\left[E+\dfrac{m_{z}\hbar\omega_{1}}{2}-\left(n_{2}+\dfrac{1}{2}\right)\omega_{2}\right]u
	\end{equation}
	\quad\quad 仍然使用幂级数法,找到渐进趋势后得到递推关系
	\begin{equation}
		\dfrac{c_{j+2}}{c_{j}}=\dfrac{\hbar\omega_{1}}{4E_{r}}\dfrac{2j+2\left|m_{z}\right|+2-\dfrac{4E_{r}}{\hbar\omega_{1}}}{\left(j+2\right)\left(j+2l+3\right)},c_{1}=0
	\end{equation}
	\quad\quad 因此得到能级
	\begin{equation}
		E=\left(n_{1}+\dfrac{1}{2}\right)\omega_{1}+\left(n_{2}+\dfrac{1}{2}\right)\omega_{2}
	\end{equation}
	\section{对称性和守恒律}
	在这一部分中为了区分作用与位置表象下波函数的算符与作用于态的算符,给所有作用于波函数的算符加上下标$p$。\par 
	先定义几个基本算符:\par
	位置平移
	\begin{align}
		\hat{T}_{p}(a)\psi(x)&=\psi(x-a)\\
		&=\sum_{n=0}^{+\infty}\dfrac{(-a)^{n}}{n!}\dfrac{\mathrm{d}^{n}}{\mathrm{d}x^{n}}\psi(x)\\
		&=\mathrm{e}^{-\mathrm{i}\frac{a}{\hbar}\hat{p}_{p}}\psi(x)
	\end{align}
	\quad\quad 动量是平移算符的生成元(由于动量算符是厄米的,平移算符是幺正的)
	\begin{equation}
		\hat{T}_{p}(a)=\mathrm{e}^{-\mathrm{i}\frac{a}{\hbar}\hat{p}_{p}}
	\end{equation}
	\quad\quad 宇称(很明显宇称算符也是幺正的)
	\begin{equation}
		\hat{\Pi}_{p}\psi(x)=-\psi(x)
	\end{equation}
	\quad\quad 转动
	\begin{equation}
		\hat{R_{z}}_{p}(\varphi)\psi(r,\theta,\phi)=\psi(r,\theta,\phi-\varphi)
	\end{equation}
	\quad\quad 可以将其推广到三维空间(角动量算符是厄米的,转动算符也是幺正算符)
	\begin{equation}
		\hat{R_{\boldsymbol{n}}}_{p}(\varphi)=\mathrm{e}^{-\mathrm{i}\frac{\varphi}{\hbar}\boldsymbol{n}\cdot\boldsymbol{\hat{L}}}
	\end{equation}
	\quad\quad 哈密顿量不含时情况下的时间平移(可以直接类比位置平移写出)
	\begin{align}
		\hat{U}_{p}(t)&=\mathrm{e}^{-\mathrm{i}\frac{t}{\hbar}\hat{H}_{p}}\\
		\hat{U}_{p}(t)\Psi(x,0)&=\Psi(x,t)
	\end{align}
	\quad\quad 实际上除了平移波函数,还可以改变算符。对于可观测量两种结果应该相同,即
	\begin{equation}
		\left<\psi\left|\hat{Q}^{\prime}\right|\psi\right>=\left<\hat{N}\psi\left|\hat{Q}\right|\hat{N}\psi\right>=\left<\psi\left|\hat{N}^{\dagger}\hat{Q}\hat{N}\right|\psi\right>
	\end{equation}
	\quad\quad 因此
	\begin{equation}
		\hat{Q}^{\prime}=\hat{N}^{\dagger}\hat{Q}\hat{N}
	\end{equation}
	\quad\quad 如果
	\begin{align}
		&\hat{H}=\hat{N}^{\dagger}\hat{H}\hat{N}\\
		&\left[\hat{H},\hat{N}\right]=0
	\end{align}
	\quad\quad 则称其具有相应对称性。这种对称性可以是连续的,也可以是离散的。有几个经典例子。\par 
	布洛赫定理。如果系统具有离散平移对称性,可以找到一组本征态使得
	\begin{equation}
		\hat{H}_{p}\psi=E\psi,\hat{T}_{p}\psi=\lambda\psi
	\end{equation}
	\quad\quad 这直接导致
	\begin{equation}
		\psi(x-a)=\mathrm{e}^{-\mathrm{i}qa}\psi(x)
	\end{equation}
	\quad\quad 一种更具有启发性的写法是
	\begin{equation}
		\psi(x)=\mathrm{e}^{\mathrm{i}qx}u(x)=\mathrm{e}^{\mathrm{i}qx}u(x+a)
	\end{equation}
	\quad\quad 关于连续对称性的另一个启发来自于哈密顿力学。对于可观测量$Q$
	\begin{equation}
		\dfrac{\mathrm{d}\left<Q\right>}{\mathrm{d}t}=\left<\dfrac{\partial\hat{Q}}{\partial t}\right>+\dfrac{\mathrm{i}}{\hbar}\left<\left[\hat{H},\hat{Q}\right]\right>
	\end{equation}
	\quad\quad 以连续平移对称性为例,无穷小平移算符
	\begin{equation}
		\hat{T}_{p}(\varepsilon)\approx1-\mathrm{i}\dfrac{\varepsilon}{\hbar}\hat{p}_{p}
	\end{equation}
	\quad\quad 因此动量算符与哈密顿算符对易,系统动量守恒。\par 
	最后一个例子是对称性引起的能级简并。一个直观的想法是对于任意一个本征函数,总可以通过对称变换构造出另一个波函数。但是这个想法是不正确的,因为变换后的波函数可能和原来的相同或仅仅只差常数因子。事实上,如果只有一个对称算符,这种对称性不会导致简并,因为总可以找到它与哈密顿算符的共同本征态。然而,如果有多个对称算符,且存在至少两个对称算符不对易,那么就不存在它们的共同本征态,此时必然出现简并。例如在三维谐振子势中,由于角动量算符之间不对易导致能级简并。\par 
	对称性的另一部分内容涉及选择定则,将在之后的部分讨论。
	\section{角动量耦合}
	考虑两个属于不同自由度角动量的耦合,我们想研究总的角动量。由于并不限定这里的角动量是轨道角动量还是自旋角动量,我们必须模仿之前对一般角动量的讨论。以两个自旋角动量为例(数学结构是矩阵的直积)
	\begin{align}
		\boldsymbol{S}&=\boldsymbol{S}^{(1)}+\boldsymbol{S}^{(2)}\\
		S_{\pm}&=S^{(1)}_{\pm}+S^{(2)}_{\pm}
	\end{align}
	\quad\quad 之后的讨论与此前完全一致,但是目前不是很容易看出总角动量的取值上下界是什么。可以这样考虑:假设在$S_{\pm}$作用后结果是$0$,则作用前$S_{z}$的本征值绝对值为$S_{z}^{(1)}$与$S_{z}^{(2)}$最大本征值之和(否则原态矢量一定存在作用后不为$0$的成分),而角动量数学结构要求$s$在$\left|s_{1}-s_{2}\right|$到$s_{1}+s_{2}$之间间隔为$1$地取值。\par 
	下面来考虑两个自旋$\dfrac{1}{2}$粒子的耦合。很容易构造出$s=1$的三重态
	\begin{align}
		&\left|1,1\right>=\left|\uparrow_{1},\uparrow_{2}\right>\\
		&\left|1,0\right>=\dfrac{1}{\sqrt{2}}\left(\left|\uparrow_{1},\downarrow_{2}\right>+\left|\downarrow_{1},\uparrow_{2}\right>\right)\\
		&\left|1,-1\right>=\left|\downarrow_{1},\downarrow_{2}\right>
	\end{align}
	\quad\quad 与$s=0$的单态
	\begin{equation}
		\left|0,0\right>=\dfrac{1}{\sqrt{2}}\left(\left|\uparrow_{1},\downarrow_{2}\right>-\left|\downarrow_{1},\uparrow_{2}\right>\right)
	\end{equation}
	\quad\quad 容易验证这几个态都是正交归一的。首先显然有
	\begin{equation}
		\left<\uparrow|\downarrow\right>=0
	\end{equation}
	\quad\quad 接下来举一个例子。直和的性质告诉我们(为了直观不得不放弃符号的严谨性)
	\begin{align}
		\left<1,1|0,0\right>&\sim\dfrac{1}{\sqrt{2}}\left(\left<\uparrow_{2},\uparrow_{1}|\uparrow_{1},\downarrow_{2}\right>-\left<\uparrow_{2},\uparrow_{1}|\downarrow_{1},\uparrow_{2}\right>\right)\\
		&\sim\dfrac{1}{\sqrt{2}}\left<\uparrow_{2}|\downarrow_{2}\right>\\
		&=0
	\end{align}
	\quad\quad 现在回到一般情况的讨论。由升降算符技巧可以发现总自旋态可以表示为复合态的线性组合
	\begin{equation}
		\left|s,m\right>=\sum_{m_{1}+m_{2}=m}C_{m_{1}m_{2}m}^{s_{1}s_{2}s}\left|s_{1},s_{2},m_{1},m_{2}\right>
	\end{equation}
	\quad\quad 其中展开系数$C_{m_{1}m_{2}m}^{s_{1}s_{2}s}$称为CG系数。一个有趣的观察是如果我们先列出所有的线性组合,反解出复合态用总自旋态展开的线性组合,然后挑选出其中$m$相同的所有线性组合排成矩阵,那么这两个矩阵都是幺正矩阵(因为不同总自旋态相互正交)。这告诉我们其实反向展开时不需要重新解方程,即
	\begin{equation}
		\left|s_{1},s_{2},m_{1},m_{2}\right>=\sum_{|s_{1}-s_{2}|\le s\le s_{1}+s_{2}}C_{m_{1}m_{2}m}^{s_{1}s_{2}s}\left|s,m_{1}+m_{2}\right>
	\end{equation}
	\section{全同粒子}
	接下来先讨论全同粒子。在经典情形下,区分两个粒子是容易的:只需要做标记或者两个粒子本身处在不同区域。而在量子力学中同种微观粒子都是完全相同的,如果两个相同微观粒子的波函数有交叠,实际上没有办法在不改变其状态的情况下加以区分。\par 
	在非相对论量子力学中,需要引入公理:自旋为整数的粒子为玻色子,自旋为半整数的粒子为费米子。如果粒子是无相互作用的,玻色子系统的波函数是交换对称的,费米子系统的波函数是交换反对称的。在这种构造下就没有办法区分每个粒子究竟处于哪一种状态。
	\section{选择定则}
	宇称算符的选择定则。分别考虑具有奇偶宇称算符的选择定则
	\begin{align}
		\Big{<}n^{'}l^{'}m^{'}\Big{|}\hat{\boldsymbol{p}}\Big{|}nlm\Big{>}&=-\Big{<}n^{'}l^{'}m^{'}\Big{|}\hat{\Pi}^{\dagger}\hat{\boldsymbol{p}}\hat{\Pi}\Big{|}nlm\Big{>}=(-1)^{l+l^{'}+1}\Big{<}n^{'}l^{'}m^{'}\Big{|}\hat{\boldsymbol{p}}\Big{|}nlm\Big{>}\\
		\Big{<}n^{'}l^{'}m^{'}\Big{|}\hat{\boldsymbol{L}}\Big{|}nlm\Big{>}&=\Big{<}n^{'}l^{'}m^{'}\Big{|}\hat{\Pi}^{\dagger}\hat{\boldsymbol{L}}\hat{\Pi}\Big{|}nlm\Big{>}=(-1)^{l+l^{'}}\Big{<}n^{'}l^{'}m^{'}\Big{|}\hat{\boldsymbol{L}}\Big{|}nlm\Big{>}
	\end{align}
	\quad\quad 氢原子波函数的宇称完全来自于角向部分,具体来说仅和$\theta$有关。连带勒让德函数的约定使得$-1$因子与求导对宇称的影响相互抵消,从而宇称仅取决于$l$。\par 
	标量算符的选择定则。无论是真标量还是赝标量在转动变换下都不变,即
	\begin{equation}
		\left[\hat{L_{z}},\hat{f}\right]=0
	\end{equation}
	\quad\quad 将其插入两个本征态之间
	\begin{align}
		\Big{<}n^{\prime}l^{\prime}m^{\prime}\Big{|}\left[\hat{L_{z}},\hat{f}\right]\Big{|}nlm\Big{>}&=\left(m^{\prime}-m\right)\left<n^{\prime}l^{\prime}m^{\prime}\right|\hat{f}\left|nlm\right>=0\\
		\Big{<}n^{\prime}l^{\prime}m^{\prime}\Big{|}\left[\hat{\boldsymbol{L}^{2}},\hat{f}\right]\Big{|}nlm\Big{>}&=\left(l^{\prime}-l\right)\left(l^{\prime}+l+1\right)\left<n^{\prime}l^{\prime}m^{\prime}\right|\hat{f}\left|nlm\right>=0
	\end{align}
	\quad\quad 因此此矩阵元可能不为$0$仅当$\Delta l=0$且$\Delta m=0$。还可以证明在此条件下矩阵元与$m$无关
	\begin{align}
		&\Big{<}n^{\prime}lm^{\prime}\Big{|}\left[\hat{L_{\pm}},\hat{f}\right]\Big{|}nlm\Big{>}\\&=\sqrt{l\left(l+1\right)-m^{\prime}\left(m^{\prime}\mp1\right)}\left<n^{\prime}lm^{\prime}\mp1\right|\hat{f}\left|nlm\right>-\sqrt{l\left(l+1\right)-m\left(m\pm1\right)}\left<n^{\prime}lm^{\prime}\right|\hat{f}\left|nlm\pm1\right>\\&=0
	\end{align}
	\quad\quad 令$m^{\prime}=m\pm1$得到(否则上式将是平凡的)
	\begin{equation}
		\left<n^{\prime}lm\right|\hat{f}\left|nlm\right>=\left<n^{\prime}lm\pm1\right|\hat{f}\left|nlm\pm1\right>
	\end{equation}
	\quad\quad 矢量算符的选择定则。
	\section{WKB近似}
	将定态波函数设为
	\begin{equation}
		\psi(x)=A(x)\mathrm{e}^{\mathrm{i}\phi(x)}
	\end{equation}
	\quad\quad 假设$\dfrac{1}{A}\dfrac{\mathrm{d}^{2}A}{\mathrm{d}x^{2}}$相比其他部分非常小,最终近似得到
	\begin{align}
		&\left(\dfrac{\mathrm{d}\phi}{\mathrm{d}x}\right)^{2}=\dfrac{p^{2}}{\hbar^{2}}\\
		&\dfrac{\mathrm{d}}{\mathrm{d}x}\left(A^{2}\dfrac{\mathrm{d}\phi}{\mathrm{d}x}\right)=0
	\end{align}
	\quad\quad 在经典区域的解为
	\begin{equation}
		\psi(x)\approx\dfrac{C}{\sqrt{p(x)}}\mathrm{e}^{\pm\frac{\mathrm{i}}{\hbar}\int^{x}p(x^{\prime})\mathrm{d}x^{\prime}}
	\end{equation}
	\quad\quad 在非经典区域的解为
	\begin{equation}
		\psi(x)\approx\dfrac{C}{\sqrt{\left|p(x)\right|}}\mathrm{e}^{\pm\frac{1}{\hbar}\int^{x}\left|p(x^{\prime})\right|\mathrm{d}x^{\prime}}
	\end{equation}
	\quad\quad 这样的近似在$p$较小时是不合理的,为此我们需要找到连接条件。在经典动量为$0$的区域将势能作线性近似,薛定谔方程变为
	\begin{equation}
		\dfrac{\mathrm{d}^{2}\psi}{\mathrm{d}x^{2}}=\dfrac{2mV^{\prime}(0)}{\hbar^{2}}x\psi
	\end{equation}
	\quad\quad 作换元$\alpha=\left(\dfrac{2mV^{\prime}(0)}{\hbar^{2}}\right)^{\frac{1}{3}}$,$z=\alpha x$
	\begin{equation}
		\dfrac{\mathrm{d}^{2}\psi}{\mathrm{d}z^{2}}=z\psi
	\end{equation}
	\quad\quad 这是艾里方程,但是我们目前并不关心具体的级数解,只需要找到渐进形式。这个方程的形式提示我们可以将试探解设为
	\begin{equation}
		u(z)=z^{\beta}\mathrm{e}^{\lambda z^{\gamma}}
	\end{equation}
	\quad\quad 从原方程可以看出来$\gamma>1$,因为它增长地比指数函数更快。带入得到\footnote{左侧按量级从大到小排列,实际确定系数时必须舍去最小的一项}
	\begin{equation}
		\lambda^{2}\gamma^{2}z^{\beta+2\gamma-2}+\lambda\gamma\left(2\beta+\gamma-1\right)z^{\beta+\gamma-2}+\beta\left(\beta-1\right)z^{\beta-2}=z^{\beta+1}
	\end{equation}
	\quad\quad 因此
	\begin{align}
		\beta&=-\dfrac{1}{4}\\
		\lambda&=\pm\dfrac{2}{3}\\
		\gamma&=\dfrac{3}{2}
	\end{align}
	\quad\quad 现在的近似解在两端都适用,但是不能保证存在一个原方程的解使得其两端的渐近展开都与试探解相同。现在必须切换到其他方法。我们先尝试用傅里叶变换找一个平方可积解,可以使用傅里叶变换
	\begin{align}
		&\psi(z)\Longleftrightarrow \xi(k)\\
		&\dfrac{\mathrm{d}\xi}{\mathrm{d}k}-\mathrm{i}k^{2}\xi=0
	\end{align}
	\quad\quad 解这个方程很容易得到一个解
	\begin{equation}
		A_{i}(z)=\dfrac{1}{\pi}\int_{0}^{+\infty}\cos\left(\dfrac{1}{3}k^{3}+kz\right)\mathrm{d}k
	\end{equation}
	\quad\quad 使用鞍点法计算出渐近展开
	\begin{equation}
		A_{i}(z)\approx\left\{\begin{matrix}
		\dfrac{\mathrm{e}^{-\frac{2}{3}z^{\frac{3}{2}}}}{2\sqrt{\pi}z^{\frac{1}{4}}},z\gg0	\\\dfrac{\sin\left(\frac{2}{3}\left(-z\right)^{\frac{3}{2}}+\frac{\pi}{4}\right)}{\sqrt{\pi}\left(-z\right)^{\frac{1}{4}}},z\ll0
		\end{matrix}\right.
	\end{equation}
	\quad\quad 利用原方程构造出另一个解
	\begin{equation}
		B_{i}(z)=\dfrac{1}{2\pi}A_{i}(z)\int_{0}^{z}\dfrac{\mathrm{d}z}{A_{i}^{2}(z)}
	\end{equation}
	\quad\quad 我们只关心渐近展开,可以直接计算积分\footnote{直接这样积分结果正确的原因是积分的主要区间恰好就是远端部分}
	\begin{equation}
		B_{i}(z)\approx\left\{\begin{matrix}
			\dfrac{\mathrm{e}^{\frac{2}{3}z^{\frac{3}{2}}}}{2\sqrt{\pi}z^{\frac{1}{4}}},z\gg0	\\\dfrac{\cos\left(\frac{2}{3}\left(-z\right)^{\frac{3}{2}}+\frac{\pi}{4}\right)}{\sqrt{\pi}\left(-z\right)^{\frac{1}{4}}},z\ll0
		\end{matrix}\right.
	\end{equation}
	\quad\quad 设$\alpha>0$,先写出远端波函数
	\begin{equation}
		\psi(x)=\left\{\begin{matrix}
		\dfrac{1}{\sqrt{p(x)}}\left[C\mathrm{e}^{-\frac{\mathrm{i}}{\hbar}\int_{0}^{x}p(x^{\prime})\mathrm{d}x^{\prime}}+D\mathrm{e}^{\frac{\mathrm{i}}{\hbar}\int_{0}^{x}p(x^{\prime})\mathrm{d}x^{\prime}}\right],x<0	\\\dfrac{1}{\sqrt{|p(x)|}}\left[E\mathrm{e}^{-\frac{1}{\hbar}\int_{0}^{x}|p(x^{\prime})|\mathrm{d}x^{\prime}}+F\mathrm{e}^{\frac{1}{\hbar}\int_{0}^{x}|p(x^{\prime})|\mathrm{d}x^{\prime}}\right],x>0
		\end{matrix}\right.
	\end{equation}
	\quad\quad 经过计算得到
	\begin{equation}
		\psi(x)=\left\{\begin{matrix}
			\dfrac{1}{\hbar^{\frac{1}{2}}\alpha^{\frac{1}{2}}(-\alpha x)^{\frac{1}{4}}}\left[C\mathrm{e}^{\mathrm{i}\frac{2}{3}(-\alpha x)^{\frac{3}{2}}}+D\mathrm{e}^{-\mathrm{i}\frac{2}{3}(-\alpha x)^{\frac{3}{2}}}\right],x<0	\\\dfrac{1}{\hbar^{\frac{1}{2}}\alpha^{\frac{1}{2}}(\alpha x)^{\frac{1}{4}}}\left[E\mathrm{e}^{-\frac{2}{3}(\alpha x)^{\frac{3}{2}}}+F\mathrm{e}^{\frac{2}{3}(\alpha x)^{\frac{3}{2}}}\right],x>0
		\end{matrix}\right.
	\end{equation}
	\quad\quad 比较得到
	\begin{equation}
		\left\{\begin{matrix}
		2E=C\mathrm{e}^{\mathrm{i}\frac{\pi}{4}}+D\mathrm{e}^{-\mathrm{i}\frac{\pi}{4}}	\\2F=C\mathrm{e}^{-\mathrm{i}\frac{\pi}{4}}+D\mathrm{e}^{\mathrm{i}\frac{\pi}{4}}
		\end{matrix}\right.
	\end{equation}
	\section{散射}
	一般来说相对于柱函数,球函数是我们更熟悉的。现在先补充这部分内容。在将亥姆霍兹方程在柱坐标下分离变量得到
	\begin{equation}
		\dfrac{\mathrm{d}^{2}R}{\mathrm{d}r^{2}}+\dfrac{1}{r}\dfrac{\mathrm{d}R}{\mathrm{d}r}+\left(k^{2}-\lambda-\dfrac{m}{r^{2}}\right)R=0
	\end{equation}
	\quad\quad 作换元$x=\sqrt{k^{2}-\lambda}$,$m=\nu^{2}$得到贝塞尔方程
	\begin{equation}
		\dfrac{\mathrm{d}^{2}R}{\mathrm{d}x^{2}}+\dfrac{1}{x}\dfrac{\mathrm{d}R}{\mathrm{d}x}+\left(1-\dfrac{\nu^{2}}{x^{2}}\right)R=0
	\end{equation}
	\quad\quad 当$\nu$不是正整数时方程的解为
	\begin{equation}
		\mathrm{J}_{\pm\nu}=\sum_{k=0}^{+\infty}\dfrac{\left(-1\right)^{k}}{k!\Gamma\left(k\pm\nu+1\right)}\left(\dfrac{x}{2}\right)^{2k\pm\nu}
	\end{equation}
	\quad\quad 否则需要定义诺伊曼函数
	\begin{equation}
		\mathrm{N}_{n}=\lim_{\nu\to n}\dfrac{\mathrm{J}_{\nu}\cos\left(\nu\pi\right)-\mathrm{J}_{-\nu}}{\sin\left(\nu\pi\right)}
	\end{equation}
	\quad\quad 诺伊曼函数在原点总是发散的,正数及非正整数阶贝塞尔函数在原点总是收敛的,其余发散。贝塞尔函数与诺伊曼函数组合成汉克尔函数
	\begin{align}
		\mathrm{H}_{\nu}^{\left(1\right)}&=\mathrm{J}_{\nu}+\mathrm{i}\mathrm{N}_{\nu}\\
		\mathrm{H}_{\nu}^{\left(2\right)}&=\mathrm{J}_{\nu}-\mathrm{i}\mathrm{N}_{\nu}
	\end{align}
	\quad\quad 当$x\to+\infty$时的渐近展开
	\begin{align}
		\mathrm{J}_{\nu}&\approx\sqrt{\dfrac{2}{\pi x}}\cos\left(x-\dfrac{\nu\pi}{2}-\dfrac{\pi}{4}\right)\\
		\mathrm{N}_{\nu}&\approx\sqrt{\dfrac{2}{\pi x}}\sin\left(x-\dfrac{\nu\pi}{2}-\dfrac{\pi}{4}\right)\\
		\mathrm{H}_{\nu}^{\left(1\right)}&\approx\sqrt{\dfrac{2}{\pi x}}\mathrm{e}^{\mathrm{i}\left(x-\frac{\nu\pi}{2}-\frac{\pi}{4}\right)}\\
		\mathrm{H}_{\nu}^{\left(2\right)}&\approx\sqrt{\dfrac{2}{\pi x}}\mathrm{e}^{-\mathrm{i}\left(x-\frac{\nu\pi}{2}-\frac{\pi}{4}\right)}
	\end{align}
	\quad\quad 散射问题往往用球坐标更加方便,亥姆霍兹方程在球坐标下分离变量得到
	\begin{equation}
		\dfrac{1}{r^{2}}\dfrac{\mathrm{d}}{\mathrm{d}r}\left(r^{2}\dfrac{\mathrm{d}R}{\mathrm{d}r}\right)+\left[k^{2}-\dfrac{l\left(l+1\right)}{r^{2}}\right]R=0
	\end{equation}
	\quad\quad 仍然作换元$x=kr$,$u=\sqrt{x}R$
	\begin{equation}
		\dfrac{\mathrm{d}^{2}u}{\mathrm{d}x^{2}}+\dfrac{1}{x}\dfrac{\mathrm{d}u}{\mathrm{d}x}+\left[1-\dfrac{\left(l+\dfrac{1}{2}\right)^{2}}{x^{2}}\right]u=0
	\end{equation}
	\quad\quad 这告诉我们可以定义球贝塞尔函数并直接得到$x\to+\infty$时的渐近展开
	\begin{align}
		&\mathrm{j}_{l}=\sqrt{\dfrac{\pi}{2x}}\mathrm{J}_{l+\frac{1}{2}}\to\dfrac{1}{x}\cos\left(x-\dfrac{l\pi}{2}-\dfrac{\pi}{2}\right)\\
		&\mathrm{n}_{l}=\sqrt{\dfrac{\pi}{2x}}\mathrm{N}_{l+\frac{1}{2}}\to\dfrac{1}{x}\sin\left(x-\dfrac{l\pi}{2}-\dfrac{\pi}{2}\right)\\
		&\mathrm{h}_{l}^{\left(1\right)}=\mathrm{j}_{l}+\mathrm{i}\mathrm{n}_{l}\to\dfrac{1}{x}\mathrm{e}^{\mathrm{i}\left(x-\frac{l\pi}{2}-\frac{\pi}{2}\right)}\\
		&\mathrm{h}_{l}^{\left(2\right)}=\mathrm{j}_{l}-\mathrm{i}\mathrm{n}_{l}\to\dfrac{1}{x}\mathrm{e}^{-\mathrm{i}\left(x-\frac{l\pi}{2}-\frac{\pi}{2}\right)}
	\end{align}
	\quad\quad 一般入射波为平面波,我们希望将其用球面波展开,即瑞利公式
	\begin{equation}
		\mathrm{e}^{\mathrm{i}kr\cos\theta}=\sum_{l=0}^{+\infty}c_{l}\mathrm{j}_{l}\left(kr\right)\mathrm{P}_{l}\left(\cos\theta\right)
	\end{equation}
	\quad\quad 因此
	\begin{equation}
		c_{l}\mathrm{j}_{l}\left(kr\right)=\dfrac{2l+1}{2}\int_{-1}^{1}\mathrm{e}^{\mathrm{i}krx}\mathrm{P}_{l}\left(x\right)\mathrm{d}x
	\end{equation}
	\quad\quad 考虑渐进形式
	\begin{equation}
		\dfrac{c_{l}}{kr}\cos\left(kr-\dfrac{l\pi}{2}-\dfrac{\pi}{2}\right)\approx\dfrac{2l+1}{2\mathrm{i}kr}\left[\mathrm{e}^{\mathrm{i}kr}-\left(-1\right)^{l}\mathrm{e}^{-\mathrm{i}kr}\right]
	\end{equation}
	\quad\quad 得到
	\begin{equation}
		c_{l}=\left(2l+1\right)\mathrm{i}^{l}
	\end{equation}
	\quad\quad 在经典力学中定义了微分散射截面
	\begin{equation}
		D(\theta)=\dfrac{\mathrm{d}\sigma}{\mathrm{d}\Omega}=\dfrac{b}{\sin\theta}\left|\dfrac{\mathrm{d}b}{\mathrm{d}\theta}\right|
	\end{equation}
	\quad\quad 先讨论分波法。现在想找到一般形式的薛定谔方程的散射解
	\begin{align}
		\psi(r,\theta)&\approx A\left[\mathrm{e}^{\mathrm{i}kz}+f(\theta)\dfrac{\mathrm{e}^{\mathrm{i}kr}}{r}\right]\\
		D(\theta)&=\left|f(\theta)\right|^{2}
	\end{align}
	\quad\quad 这样的解在远区自然是成立的。现在还需要讨论中间区。这部分可以忽略散射势但是不忽略离心势,解的形式为
	\begin{equation}
		\psi\left(r,\theta,\phi\right)=A\left[\mathrm{e}^{\mathrm{i}kz}+\sum_{l,m}C_{lm}\mathrm{h}_{l}^{\left(1\right)}\left(kr\right)\mathrm{Y}_{l}^{m}\left(\theta,\phi\right)\right]
	\end{equation}
	\quad\quad 只考虑旋转对称情况,约定将展开系数重新取为
	\begin{equation}
		\psi\left(r,\theta\right)=A\left[\mathrm{e}^{\mathrm{i}kz}+k\sum_{l=0}^{+\infty}\mathrm{i}^{l+1}\left(2l+1\right)a_{l}\mathrm{h}_{l}^{\left(1\right)}\left(kr\right)\mathrm{P}_{l}\left(\cos\theta\right)\right]
	\end{equation}
	\quad\quad 这告诉我们
	\begin{equation}
		f\left(\theta\right)=\sum_{l=0}^{+\infty}\left(2l+1\right)a_{l}\mathrm{P}_{l}\left(\cos\theta\right)
	\end{equation}
	\quad\quad 从而总散射截面
	\begin{equation}
		\sigma=4\pi\sum_{l=0}^{+\infty}\left(2l+1\right)\left|a_{l}\right|^{2}
	\end{equation}
	\quad\quad 进一步带入瑞利公式
    \begin{equation}
    	\psi\left(r,\theta\right)=A\sum_{l=0}^{+\infty}\mathrm{i}^{l}\left(2l+1\right)\left[\mathrm{j}_{l}\left(kr\right)+\mathrm{i}ka_{l}\mathrm{h}_{l}^{\left(1\right)}\left(kr\right)\right]\mathrm{P}_{l}\left(\cos\theta\right)
    \end{equation}
    \quad\quad 在此前的散射问题中,我们定义过透射、反射系数,它们都归根于求解相移。考虑上式的渐进性形式
    \begin{equation}
    	\psi\left(r,\theta\right)\approx\dfrac{A}{2\mathrm{i}kr}\sum_{l=0}^{+\infty}\left(2l+1\right)\left[\left(1+2\mathrm{i}ka_{l}\right)\mathrm{e}^{\mathrm{i}kr}-\left(-1\right)^{l}\mathrm{e}^{-\mathrm{i}kr}\right]\mathrm{P}_{l}\left(\cos\theta\right)
    \end{equation}
    \quad\quad 通过薛定谔方程得到几率流守恒方程
    \begin{equation}
    	\mathrm{i}\hbar\dfrac{\partial}{\partial t}\left|\Psi\right|^{2}=-\dfrac{\hbar^{2}}{2m}\boldsymbol{\nabla}\cdot\left(\Psi^{*}\boldsymbol{\nabla}\Psi-\Psi\boldsymbol{\nabla}\Psi^{*}\right)
    \end{equation}
    \quad\quad 对于定态右侧体积分为$0$。初始的平面波分解为不同角动量本征值的状态,而散射过程角动量守恒,这说明只需要考虑单个分波即可\footnote{实际上由于这里涉及角动量量子数,对$l$求和中的每一项都是对应薛定谔方程的解,所以才能分开考虑}。我们考虑如下形式波函数
    \begin{equation}
    	\xi=\dfrac{B\mathrm{e}^{\mathrm{i}kr}+C\mathrm{e}^{-\mathrm{i}kr}}{r}
    \end{equation}
    \begin{align}
    	\lim_{r\to+\infty}r^{2}\left(\xi^{*}\boldsymbol{\nabla}\xi-xi\boldsymbol{\nabla}\xi^{*}\right)=0\to\left|B\right|^{2}=\left|C\right|^{2}
    \end{align}
    \quad\quad 因此可以引入相移因子
    \begin{align}
    	\mathrm{e}^{\mathrm{2i\delta_{l}}}&=1+2\mathrm{i}ka_{l}\\
    	a_{l}&=\dfrac{1}{k}\mathrm{e}^{\mathrm{i}\delta_{l}}\sin\delta_{l}
    \end{align}
    \quad\quad 此前各公式可以重新表示为
    \begin{align}
    	f\left(\theta\right)&=\sum_{l=0}^{+\infty}\left(2l+1\right)a_{l}\dfrac{1}{k}\mathrm{e}^{\mathrm{i}\delta_{l}}\sin\delta_{l}\mathrm{P}_{l}\left(\cos\theta\right)\\
    	\sigma&=\dfrac{4\pi}{k^{2}}\sum_{l=0}^{+\infty}\left(2l+1\right)\sin^{2}\delta_{l}
    \end{align}
    \quad\quad 散射问题还有另一种处理方法。先写出定态薛定谔方程的形式解
    \begin{align}
    	\psi\left(\boldsymbol{r}\right)=\psi_{0}\left(\boldsymbol{r}\right)-\dfrac{m}{2\pi\hbar^{2}}\iiint&\mathrm{d}^{3}\boldsymbol{r^{\prime}}\dfrac{\mathrm{e}^{\mathrm{i}k\left|\boldsymbol{r}-\boldsymbol{r^{\prime}}\right|}}{\left|\boldsymbol{r}-\boldsymbol{r^{\prime}}\right|}V\left(\boldsymbol{r^{\prime}}\right)\psi\left(\boldsymbol{r^{\prime}}\right)\\
    	\left(\boldsymbol{\nabla}^{2}+k^{2}\right)&\psi_{0}=0
    \end{align}
    \quad\quad 将这个形式解反复带入自身得到的级数成为玻恩级数,它有可能是收敛的。如果仍然认为散射势局限在一个较小的范围内,那么可以近似为
    \begin{equation}
    	\psi\left(\boldsymbol{r}\right)=\psi_{0}\left(\boldsymbol{r}\right)-\dfrac{m}{2\pi\hbar^{2}}\dfrac{\mathrm{e}^{\mathrm{i}k\left|\boldsymbol{r}\right|}}{\left|\boldsymbol{r}\right|}\iiint\mathrm{d}^{3}\boldsymbol{r^{\prime}}\mathrm{e}^{-\mathrm{i}\boldsymbol{k_{r}}\cdot\boldsymbol{r^{\prime}}}V\left(\boldsymbol{r^{\prime}}\right)\psi\left(\boldsymbol{r^{\prime}}\right)
    \end{equation}
	\section{定态微扰理论}
	使用微扰法近似求解定态薛定谔方程。先假设无微扰本征态是非简并的。用$\lambda$来标记数量级,保留到二阶项
	\begin{align}
		\hat{H}&=\hat{H^{0}}+\lambda\hat{H^{\prime}}\\
		\left|\psi_{n}\right>&=\left|\psi^{0}_{n}\right>+\lambda\left|\psi^{1}_{n}\right>+\lambda^{2}_{n}\left|\psi^{2}\right>\\
		E_{n}&=E_{n}^{0}+E_{n}^{1}+E_{n}^{2}
	\end{align}
	\quad\quad 带入定态薛定谔方程得到
	\begin{align}
		\hat{H^{0}}\left|\psi_{n}^{0}\right>&=E_{n}^{0}\left|\psi_{n}^{0}\right>\\
		\hat{H^{0}}\left|\psi_{n}^{1}\right>+\hat{H^{\prime}}\left|\psi_{n}^{0}\right>&=E_{n}^{0}\left|\psi_{n}^{1}\right>+E_{n}^{1}\left|\psi_{n}^{0}\right>\\
		\hat{H^{0}}\left|\psi_{n}^{2}\right>+\hat{H^{\prime}}\left|\psi_{n}^{1}\right>&=E_{n}^{0}\left|\psi_{n}^{2}\right>+E_{n}^{1}\left|\psi_{n}^{1}\right>+E_{n}^{2}\left|\psi_{n}^{0}\right>
	\end{align}
	\quad\quad 取内积得到
	\begin{align}
		E_{n}^{1}&=\left<\psi_{n}^{0}\right|\hat{H^{\prime}}\left|\psi_{n}^{0}\right>\\
		\left|\psi_{n}^{1}\right>&=\sum_{m\ne n}\dfrac{\left<\psi_{m}^{0}\right|\hat{H^{\prime}}\left|\psi_{n}^{0}\right>}{E_{n}^{0}-E_{m}^{0}}\left|\psi_{m}^{0}\right>\\
		E_{n}^{2}&=\sum_{m\ne n}\dfrac{\left|\left<\psi_{m}^{0}\right|\hat{H^{\prime}}\left|\psi_{n}^{0}\right>\right|^{2}}{E_{n}^{0}-E_{m}^{0}}
	\end{align}
	\quad\quad 现在来考虑包含简并的情形。简并态可以进行正交化,我们希望正交化得到的本征态恰好就是逐步撤去微扰时本征态的极限。假设本征态已经全部正交化,现在来定出展开系数。通过取内积可以将其变为本征值问题,而本征矢代表着具体的组合方式\footnote{对于简并情情形,有一种观点是简并情形导致修正求和项中存在分母为$0$的部分,所以结果发散,必须变为本征值问题。至少我认为这种说法是不正确的。最初我们在计算一阶能量修正时取内积的过程与简并情形实际上没有任何区别,但是由于已经假设了非简并,只需要解决一阶矩阵的本征值问题。换言之,其实我们始终在构造本征值问题,只不过此前没有刻意指出。现在说明到底为什么必须找到简并态的特定组合方式:因为如果不这样组合,一阶近似方程就是自相矛盾的。实际上,分母为$0$的问题从来就不存在,也不是引入简并微扰理论的理由,求解本征值问题本身只是计算修正中不得不做的一步。在求解完本征值问题后构造出的本征态自然就是取微扰为$0$极限时的结果,因为修正大小与哈密顿量变化正相关,二者同时存在,同时消失。}
	\begin{align}
		\overleftrightarrow{W}\alpha&=E^{1}\alpha\\
		W_{mn}&=\left<\psi_{m}\right|\hat{H^{\prime}}\left|\psi_{n}\right>
	\end{align}
	\quad\quad 这种方法求解起来较为麻烦,可以使用如下定理:如果存在厄米算符$\hat{A}$与微扰后哈密顿量对易,且简并本征态(已经正交化)是$\hat{A}$属于不同特征值的本征态,那么这些本征态就是我们要找的。以二重简并为例,假设
	\begin{align}
		\hat{H^{0}}\left|\psi_{a}^{0}\right>&=\hat{H^{0}}\left|\psi_{b}^{0}\right>=E^{0}\\
		\hat{A}\left|\psi_{a}^{0}\right>&=\mu\left|\psi_{a}^{0}\right>\\
		\hat{A}\left|\psi_{b}^{0}\right>&=\nu\left|\psi_{b}^{0}\right>
	\end{align}
	\quad\quad $\hat{A}$与哈密顿量有共同本征态
	\begin{equation}
		\hat{A}\left|\xi\right>=\gamma\left|\xi\right>
	\end{equation}
	\quad\quad 由$\hat{A}$的厄米性得到
	\begin{align}
		(\gamma-\mu)\left<\psi_{a}^{0}\big{|}\xi\right>&=0\\
		(\gamma-\nu)\left<\psi_{b}^{0}\big{|}\xi\right>&=0
	\end{align}
	\quad\quad 如果取微扰为$0$的极限,$\gamma$必须是$\mu$和$\nu$中的一个,否则两个内积都是$0$,这个共同本征态的极限是$0$,不符合要求。这就说明内积有一个是$0$,有一个不是$0$,这就说明极限是简并本征态中的一个。由于已经完成正交化,这一组简并本征态满足要求\footnote{实际上往往能够通过对称性找到$\hat{A}$并通过相应对称性适当组合本征函数}。
	\section{含时微扰理论}
	
\end{document}