\documentclass[12pt, a4paper, oneside]{ctexart}
\usepackage{amsmath, amsthm, amssymb, graphicx}
\usepackage[bookmarks=true, colorlinks, citecolor=blue, linkcolor=black]{hyperref}
\usepackage{fancyhdr}
\usepackage{abstract}
\usepackage{xcolor}
\usepackage{geometry}
\geometry{a4paper,left=0.6in,right=0.6in,top=1.2in,bottom=1.2in}
\setlength\columnsep{0.5in}
\columnseprule=1pt
\pagestyle{fancy}
\fancyhf{}
\fancyhead[R]{\thepage} 
\fancyhead[C]{笔记} 
% 导言区
\title{\Huge 笔记}
\date{\today}
\setlength{\columnsep}{1in}
\newcommand{\blankspace}{\rule{2cm}{0.15mm}}
\begin{document}
	\maketitle
	\section{波函数}
	薛定谔方程
	\begin{equation}
		\mathrm{i}\hbar\dfrac{\partial\Psi}{\partial t}=-\dfrac{\hbar^{2}}{2m}\boldsymbol{\nabla}^{2}\Psi+V\Psi
	\end{equation}
	\quad\quad 或者写为
	\begin{equation}
		\mathrm{i}\hbar\dfrac{\partial\Psi}{\partial t}=\hat{H}\Psi
	\end{equation}
	\quad\quad 波恩的统计诠释认为波函数的模方表示发现粒子位于空间中某处的概率密度,即在$t$时刻在区域$\Omega$发现粒子的概率是
	\begin{equation}
		\iiint_{\Omega}\mathrm{d}^{3}x\left|\Psi\left(\boldsymbol{r},t\right)\right|^{2}
	\end{equation}
	\quad\quad 这种统计诠释引入了一种不确定性,即使已经知道了一个粒子的波函数,仍无法通过一次简单的位置测量来验证这个结果,因此波函数只能提供可能结果的统计信息。这种不确定性引发了一个问题:如果测量发现粒子在$C$处,那么在测量开始的前一刻粒子在哪里?有三种看法:\par 
	1.哥本哈根学派的观点:是测量导致粒子“位于$C$”处,换言之,观测本身不可避免地扰动了被观测量,导致波函数坍缩、观测到对应结果。这是被最广泛接受的观点,但它存在一个明显的问题:如果测量过程完全是一个真实存在的物理过程,它应该满足薛定谔方程,不应引起波函数坍缩;如果不是,那么和真实物理过程的界限在哪里?\par 
	2.现实主义学派的观点:粒子就在$C$点。这是爱因斯坦等人的观点。他们认为意味量子力学是不完备的,因为它没有办法告诉我们这一点。粒子的位置不是不可确定的,这说明除了$\Psi$之外还需要提供附加信息,称为隐变量。这一观点会导致更加令人不安的问题,考虑EPRB佯谬(EPR佯谬的一个简化版本):一个中性$\pi^{0}$介子的衰变过程
	\begin{equation}
		\pi^{0}\to\mathrm{e}^{-}+\mathrm{e}^{+}
	\end{equation}
	\quad\quad 由于$\pi^{0}$介子的自旋为$0$,由角动量守恒,正负电子的自旋必须相反,因此只需要测量电子的自旋,就立刻能够得知正电子自旋的测量结果,无论两个电子相距多远。假定任何作用的传播速度不能超过真空光速(定域性原则)。根据哥本哈根诠释,正负电子自旋的“产生”都是瞬间发生的,这明显存在矛盾;而现实主义学派的观点则避免了这个问题,因为二者的自旋从一开始就是确定的。\par 
	因此爱因斯坦等人声称为了完整描述物理实际需要额外引入隐变量$\lambda$。现在设置两个方向任意的探测器,考虑自旋乘积的平均值,假设两个探测器夹角为$\theta$,根据量子力学测量结果应该为
	\begin{equation}
		P(\boldsymbol{e_{1}},\boldsymbol{e_{2}})=-\dfrac{\hbar^{2}}{4}\cos\theta
	\end{equation}
	\quad\quad 根据现实主义学派的观点,如果用函数$A(\boldsymbol{e_{1}},\lambda)$、$B(\boldsymbol{e_{2}},\lambda)$表示正负电子的测量结果,这两个结果都只能取$\pm \dfrac{\hbar}{2}$,并且要满足角动量守恒
	\begin{equation}
		A(\boldsymbol{e},\lambda)+B(\boldsymbol{e},\lambda)=0
	\end{equation}
	\quad\quad 给隐变量引入一个概率密度,那么
	\begin{align}
		P(\boldsymbol{e_{1}},\boldsymbol{e_{2}})&=\int\mathrm{d}\lambda\rho(\lambda)A(\boldsymbol{e_{1}},\lambda)B(\boldsymbol{e_{2}},\lambda)\\
		&=-\int\mathrm{d}\lambda\rho(\lambda)A(\boldsymbol{e_{1}},\lambda)A(\boldsymbol{e_{2}},\lambda)\\
	\end{align}
	\quad\quad 这导致
	\begin{align}
		\left|P(\boldsymbol{e_{1}},\boldsymbol{e_{2}})-P(\boldsymbol{e_{1}},\boldsymbol{e_{3}})\right|&=\left|\int\mathrm{d}\lambda\rho(\lambda)\left[A(\boldsymbol{e_{1}},\lambda)A(\boldsymbol{e_{2}},\lambda)-A(\boldsymbol{e_{1}},\lambda)A(\boldsymbol{e_{3}},\lambda)\right]\right|\\
		&=\left|\int\mathrm{d}\lambda\rho(\lambda)\left[1-\dfrac{4}{\hbar^{2}}A(\boldsymbol{e_{2}},\lambda)A(\boldsymbol{e_{3}},\lambda)\right]A(\boldsymbol{e_{1}},\lambda)A(\boldsymbol{e_{2}},\lambda)\right|\\
		&\le\int\mathrm{d}\lambda\rho(\lambda)\left[\dfrac{\hbar^{2}}{4}-A(\boldsymbol{e_{2}},\lambda)A(\boldsymbol{e_{3}},\lambda)\right]\\
		&=\dfrac{\hbar^{2}}{4}+P(\boldsymbol{e_{2}},\boldsymbol{e_{3}})
	\end{align}
	\quad\quad 上式称为贝尔不等式,容易验证它与量子力学的结果是不相容的,而实验结果也证明贝尔不等式不成立,这说明定域性隐变量理论是错误的。\par 
	3.这个问题没有意义,因为测量前的状态不可能通过测量得到,换言之,得到的结果一定不是“测量前的”。\par 
	总而言之,这是量子力学中一个敏感的问题,至今没有得到非常圆满的解释。在之后将采用大多数人接受的哥本哈根学派的观点。\par 
	波函数必须满足归一化关系
	\begin{equation}
		\iiint\mathrm{d}^{3}x\left|\Psi(\boldsymbol{x},t)\right|^{2}=1
	\end{equation}
	\quad\quad 为了验证它总是归一化的,进行如下计算
	\begin{align}
		\dfrac{\mathrm{d}}{\mathrm{d}t}\iiint\mathrm{d}^{3}x\left|\Psi(\boldsymbol{x},t)\right|^{2}&=\iiint\mathrm{d}^{3}x\dfrac{\partial}{\partial t}\left|\Psi(\boldsymbol{x},t)\right|^{2}\\
		&=\mathrm{i}\dfrac{\hbar}{2m}\iiint\mathrm{d}^{3}x\left(\Psi^{*}\boldsymbol{\nabla}^{2}\Psi-\Psi\boldsymbol{\nabla}^{2}\Psi^{*}\right)\\
		&=0
	\end{align}
	\quad\quad 因此只要在某个时刻归一化了波函数,以后它都是归一化的。现在考虑动量
	\begin{equation}
		\left<\boldsymbol{p}\right>=m\dfrac{\mathrm{d}\left<x\right>}{\mathrm{d}t}=\iiint\mathrm{d}^{3}x\Psi^{*}\left(-\mathrm{i}\hbar\boldsymbol{\nabla}\right)\Psi
	\end{equation}
	\quad\quad 也可以写成(其实不应该混淆态和波函数,波函数实际上是态矢量在坐标表象下的投影。但是,如果始终默认选取坐标表象的话,是否加以区分影响不大)
	\begin{equation}
		\left<\boldsymbol{p}\right>=\left<\Psi|\boldsymbol{\hat{p}}\Psi\right>
	\end{equation}
	\quad\quad 此外还可以导出
	\begin{equation}
		\dfrac{\mathrm{d}\left<\boldsymbol{p}\right>}{\mathrm{d}t}=\left<-\boldsymbol{\nabla}V\right>
	\end{equation}
	\quad\quad 对于可观测量$\boldsymbol{Q}$,它要求
	\begin{equation}
		\left<\Psi\Big{|}\boldsymbol{\hat{Q}}\Psi\right>=\left<\boldsymbol{\hat{Q}}\Psi\big{|}\Psi\right>
	\end{equation}
	\quad\quad 这种算符与它的厄米共轭相等,称为厄米算符。根据测量公设,对于定态,测量结果将是其本征值$q$,此外还需要假设厄米算符的本征函数系是完备的。现在可以引入广义统计诠释:(为方便起见假设是离散谱)系统波函数可用本征函数展开
	\begin{equation}
		\Psi=c_{n}f_{n}
	\end{equation}
	\quad\quad 其中
	\begin{equation}
		c_{n}=\left<f_{n}|\Psi\right>
	\end{equation}
	\quad\quad 在一次测量中得到本征值$q_{k}$的概率是$|c_{n}|^{2}$,测量之后波函数坍缩至相应本征态。\par 
	现在还需要对易关系。它的灵感来自于泊松括号。定义
	\begin{equation}
		\left[\hat{A},\hat{B}\right]=\hat{A}\hat{B}-\hat{B}\hat{A}
	\end{equation}
	\quad\quad 可以计算一些典型的对易关系,例如
	\begin{align}
		\left[f(x),\hat{p}\right]&=\mathrm{i}\hbar\dfrac{\mathrm{d}f}{\mathrm{d}x}\\
		\left[r_{\alpha},\hat{p_{\beta}}\right]&=\mathrm{i}\hbar\delta_{\alpha\beta}\\
		\left[\hat{L_{\alpha}},\hat{L_{\beta}}\right]&=\mathrm{i}\hbar \epsilon_{\alpha\beta\gamma}\hat{L_{\gamma}}\\
		\left[\hat{L_{\alpha}},x_{\beta}\right]&=\mathrm{i}\hbar \epsilon_{\alpha\beta\gamma}x_{\gamma}\\
		\left[\hat{L_{\alpha}},\hat{p_{\beta}}\right]&=\mathrm{i}\hbar \epsilon_{\alpha\beta\gamma}\hat{p_{\gamma}}\\
	\end{align}
	\section{定态薛定谔方程}
	如果势能不含时,可以将薛定谔方程分离变量得到含时摇摆因子
	\begin{equation}
		\mathrm{e}^{-\mathrm{i}\frac{E}{\hbar}t}
	\end{equation}
	\quad\quad 与定态薛定谔方程
	\begin{equation}
		\hat{H}\psi=E\psi
	\end{equation}
	\quad\quad 对于谐振子模型,构造升降算符(两个升降算符互为厄米共轭算符)
	\begin{align}
		\hat{a_{\pm}}&=\dfrac{1}{\sqrt{2m\hbar\omega}}\left(m\omega x\mp\mathrm{i}\hat{p}\right)\\
		\hat{H}&=\hbar\omega\left(\hat{a_{\mp}}\hat{a_{\pm}}\mp\dfrac{1}{2}\right)
	\end{align}
	\quad\quad 假设现在已经找到了本征态$\psi$,容易推出
	\begin{align}
		\hat{H}\left(\hat{a_{\pm}\psi}\right)&=\hbar\omega\left(\hat{a_{\pm}}\hat{a_{\mp}}\hat{a_{\pm}}\mp\dfrac{1}{2}\hat{a_{\pm}}\right)\psi\\
		&=\hat{a_{\pm}}\left(\hat{H}\pm\hbar\omega\right)\psi\\
		&=\left(E\pm\hbar\omega\right)\hat{a_{\pm}}\psi
	\end{align}
	\quad\quad 由于能量不可能一直降低,一定存在基态满足
	\begin{align}
		\hat{a_{-}}\psi_{0}=0\to\dfrac{\mathrm{d}\psi_{0}}{\mathrm{d}x}+\dfrac{m\omega}{\hbar}x\psi_{0}=0
	\end{align}
	\quad\quad 可以看出解为
	\begin{align}
		\psi_{0}&=\left(\dfrac{m\omega}{\pi\hbar}\right)^{\frac{1}{4}}\mathrm{e}^{-\frac{m\omega}{2\hbar}x^{2}}\\
		E_{0}&=\dfrac{1}{2}\hbar\omega
	\end{align}
	\quad\quad 为了定出各个激发态的归一化系数,可以考虑计算
	\begin{align}
		\hat{a_{+}}\hat{a_{-}}\psi_{n}&=n\psi_{n}\\
		\hat{a_{-}}\hat{a_{+}}\psi_{n}&=(n+1)\psi_{n}\\
	\end{align}
	\quad\quad 与$\psi_{n}$取内积得到
	\begin{align}
		\hat{a_{-}}\psi_{n}&=\sqrt{n}\psi_{n}\\
		\hat{a_{+}}\psi_{n}&=\sqrt{n+1}\psi_{n}\\
	\end{align}
	\quad\quad 因此
	\begin{equation}
		\psi_{n}=\dfrac{1}{\sqrt{n}}\left(\hat{a_{+}}\right)^{n}\psi_{0}
	\end{equation}
	\quad\quad 现在转而讨论氢原子。为了简单起见,先用角动量算符表示拉普拉斯算符
	\begin{equation}
		\boldsymbol{\nabla}^{2}=\dfrac{1}{r^{2}}\dfrac{\partial}{\partial r}\left(r^{2}\dfrac{\partial}{\partial r}\right)-\dfrac{\boldsymbol{\hat{L}}^{2}}{\hbar r^{2}}
	\end{equation}
	\quad\quad 我们早已知道分离变量解的角向部分是球谐函数,满足
	\begin{equation}
		\dfrac{\boldsymbol{\hat{L}}^{2}}{\hbar}\mathrm{Y}_{l}^{m}(\theta,\phi)=l(l+1)\mathrm{Y}_{l}^{m}(\theta,\phi)
	\end{equation}
	\quad\quad 这样很快得到了径向方程
	\begin{equation}
		-\dfrac{\hbar^{2}}{2m}\left[\dfrac{1}{r}\dfrac{\mathrm{d}}{\mathrm{d}r}\left(r^{2}\dfrac{\mathrm{d}R}{\mathrm{d}r}\right)-\dfrac{l\left(l+1\right)R}{r}\right]-\dfrac{e^{2}R}{4\pi\varepsilon_{0}}=ERr
	\end{equation}
	\quad\quad 其实这里我们可以从光学中球面波获得启发,作换元$u=Rr$,$\kappa=\dfrac{\sqrt{-2mE}}{\hbar}$这样得到
	\begin{equation}
		\dfrac{1}{\kappa^{2}}\dfrac{\mathrm{d}^{2}u}{\mathrm{d}r^{2}}=\left[1-\dfrac{me^{2}}{2\pi\varepsilon_{0}\hbar^{2}\kappa}\dfrac{1}{\kappa r}+\dfrac{l\left(l+1\right)}{\left(\kappa r\right)^{2}}\right]u
	\end{equation}
	\quad\quad 进一步令$\rho=\kappa r$,$\rho_{0}=\dfrac{me^{2}}{2\pi\varepsilon_{0}\hbar^{2}\kappa}$得到
	\begin{equation}
		\dfrac{\mathrm{d}^{2}u}{\mathrm{d}\rho^{2}}=\left[1-\dfrac{\rho_{0}}{\rho}+\dfrac{l\left(l+1\right)}{\rho^{2}}\right]u
	\end{equation}
	\quad\quad 渐进行为要求我们将解的形式设为(这是因为我们不希望讨论洛朗级数)
	\begin{equation}
		u=\rho^{l+1}\mathrm{e}^{-\rho}v
	\end{equation}
	\quad\quad 即
	\begin{equation}
		\rho\dfrac{\mathrm{d}^{2}v}{\mathrm{d}\rho^{2}}+2(l+1-\rho)\dfrac{\mathrm{d}v}{\mathrm{d}\rho}+\left[\rho_{0}-2(l+1)\right]v=0
	\end{equation}
	\quad\quad 现在可以带入幂级数解
	\begin{equation}
		v=\sum_{j=0}^{+\infty}c_{j}\rho^{j}
	\end{equation}
	\quad\quad 整理得
	\begin{equation}
		\dfrac{c_{j+1}}{c_{j}}=\dfrac{2\left(j+l+1\right)-\rho_{0}}{\left(j+1\right)\left(j+2l+2\right)}
	\end{equation}
	\quad\quad 如果有无穷多项,当$j$充分大的时候$v\sim\mathrm{e}^{2\rho}$,因此必须截断,即
	\begin{align}
		\rho_{0}&=2n\\
		N&=l-n\\
		c_{N}&=0
	\end{align}
	\quad\quad 这样得到了能级公式
	\begin{equation}
		E_{n}=-\dfrac{m}{2\hbar^{2}}\left(\dfrac{e^{2}}{4\pi\varepsilon_{0}}\right)^{2}\dfrac{1}{n^{2}}
	\end{equation}
	\quad\quad 定义玻尔半径
	\begin{align}
		a&=\dfrac{4\pi\varepsilon_{0}\hbar^{2}}{me^{2}}\\
		\rho&=\dfrac{r}{an}
	\end{align}
	\quad\quad 经过冗长的计算得到得到定态波函数
	\begin{equation}
		\psi_{nlm}=\sqrt{\left(\frac{2}{n a}\right)^{3} \frac{(n-l-1)!}{2 n(n+l)!}} \mathrm{e}^{-r / n a}\left(\frac{2 r}{n a}\right)^{l}\left[\mathrm{L}_{n-l-1}^{2 l+1}(2 r / n a)\right] \mathrm{Y}_{l}^{m}(\theta, \phi)
	\end{equation}
	\quad\quad 其中缔合拉盖尔多项式
	\begin{equation}
		\mathrm{L}^{p}_{q}(x)=\dfrac{(-1)^{p}}{(p+q)!}\dfrac{\mathrm{d}^{p}}{\mathrm{d}x^{p}}\left[\mathrm{e}^{x}\dfrac{\mathrm{d}^{p+q}}{\mathrm{d}x^{p+q}}\left(e^{-x}x^{p+q}\right)\right]
	\end{equation}
	\section{自旋}
	接下来将要讨论一个略显微妙的问题。容易计算对易关系
	\begin{equation}
		\left[\boldsymbol{\hat{L}}^{2},\hat{L_{z}}\right]=0
	\end{equation}
	\quad\quad 我们期待能够找到共同本征态
	\begin{align}
		\boldsymbol{\hat{L}}^{2}f&=\lambda f\\
		\hat{L_{z}}f&=\mu f
	\end{align}
	\quad\quad 仍然构造升降算符
	\begin{align}
		\hat{L_{\pm}}&=\hat{L_{x}}\pm\mathrm{i}\hat{L_{y}}\\
		\boldsymbol{\hat{L}}^{2}\hat{L_{\pm}}f&=\hat{L_{\pm}}\boldsymbol{\hat{L}}^{2}f=\lambda\boldsymbol{\hat{L}}^{2}\hat{L_{\pm}}f\\
		\hat{L_{z}}\hat{L_{\pm}}f&=\hat{L_{\pm}}\hat{L_{z}}f\pm\hbar\hat{L_{\pm}}f=(\lambda\pm\hbar)\hat{L_{\pm}}f
	\end{align}
	\quad\quad 由于分量的平方不可能超过总量的平方,必然有上下限
	\begin{align}
		\hat{L_{+}}f_{top}&=0\\
		\hat{L_{z}}f_{top}&=l\hbar\\
		\boldsymbol{\hat{L}}^{2}f_{top}&=\lambda\\
		\hat{L_{-}}f_{bottom}&=0\\
		\hat{L_{z}}f_{bottom}&=\overline{l}\hbar\\
		\boldsymbol{\hat{L}}^{2}f_{bottom}&=\lambda\\
	\end{align}
	\quad\quad 注意到有如下展开
	\begin{equation}
		\boldsymbol{\hat{L}}^{2}=\hat{L_{\pm}}\hat{L_{\mp}}\mp\hbar\hat{L_{z}}+\hat{L_{z}}^{2}
	\end{equation}
	\quad\quad 因此
	\begin{align}
		\lambda=\hbar^{2}l(l+1)&=\hbar^{2}\overline{l}(\overline{l}-1)\\
		\overline{l}&=-l
	\end{align}
	\quad\quad 现在我们发现最后的形式是我们熟悉的
	\begin{align}
		\boldsymbol{\hat{L}}^{2}f^{m}_{l}&=\hbar^{2}l(l+1) f^{m}_{l}\\
		\hat{L_{z}}f^{m}_{l}&=\hbar mf^{m}_{l}
	\end{align}
	\quad\quad 通过与之前类似的方法可以得到
	\begin{align}
		\hat{L_{+}}f^{m}_{l}&=\sqrt{l(l+1)-m(m+1)}f^{m+1}_{l}\\
		\hat{L_{-}}f^{m}_{l}&=\sqrt{l(l+1)-m(m-1)}f^{m-1}_{l}
	\end{align}
	\quad\quad 我们发现这里的角动量$z$分量本征值并不限定为$\hbar$的整数倍,还可以是半整数倍。这是因为这里讨论的是一般形式的角动量。之前在坐标表象下求解角动量本征值时受到波函数单值性的限制,本征值只能取整数,而接下来要讨论的自旋角动量可以取半整数。在非相对论量子力学中只能认为自旋是粒子的内禀属性,除此之外不做任何深入探讨,自然也不能像之前那样写出波函数,但仍可用态矢量表示。现在先把它当作轨道角动量理论的翻版构造对易关系
	\begin{equation}
		\left[\hat{S_{\alpha}},\hat{S_{\beta}}\right]=\mathrm{i}\hbar \epsilon_{\alpha\beta\gamma}\hat{S_{\gamma}}
	\end{equation}
	\quad\quad 同样地
	\begin{align}
		\boldsymbol{\hat{S}}^{2}\left|s,m\right>&=\hbar^{2}l(l+1)\left|s,m\right>\\
		\hat{S_{z}}\left|s,m\right>&=\hbar m\left|s,m\right>\\
		\hat{S_{+}}\left|s,m\right>&=\sqrt{s(s+1)-m(m+1)}\left|s,m+1\right>\\
		\hat{S_{-}}\left|s,m\right>&=\sqrt{s(s+1)-m(m-1)}\left|s,m-1\right>
	\end{align}
	\quad\quad 最简单的情况是$s=\dfrac{1}{2}$,共有两个本征态
	\begin{equation}
		\left|\dfrac{1}{2},\dfrac{1}{2}\right>,\left|\dfrac{1}{2},-\dfrac{1}{2}\right>
	\end{equation}
	\quad\quad 这种粒子的状态可以用旋量表示
	\begin{equation}
		\chi=\begin{pmatrix}
			a\\b
		\end{pmatrix}
		=a\chi_{+}+b\chi_{-}=a\begin{pmatrix}
			1\\0
		\end{pmatrix}+b\begin{pmatrix}
		0\\1
		\end{pmatrix}
	\end{equation}
	\quad\quad 这里其实也不应该混淆态矢量和旋量。态矢量属于希尔伯特空间,而旋量是相对特定基矢量的一组分量。目前这种情况可以用矩阵表示自旋算符(在没有歧义的情况下不区分张量、矩阵和算符,仍沿用比较直观的符号)
	\begin{align}
		\hat{S_{z}}&=\dfrac{\hbar}{2}\overleftrightarrow{\sigma_{z}}=\dfrac{\hbar}{2}\begin{pmatrix}
		1	& 0\\
		0	&-1
		\end{pmatrix}\\
		\hat{S_{x}}&=\dfrac{\hbar}{2}\overleftrightarrow{\sigma_{x}}=\dfrac{\hbar}{2}\begin{pmatrix}
			0	& 1\\
			1	& 0
		\end{pmatrix}\\
		\hat{S_{y}}&=\dfrac{\hbar}{2}\overleftrightarrow{\sigma_{y}}=\dfrac{\hbar}{2}\begin{pmatrix}
		0	& -\mathrm{i}\\
		\mathrm{i}	& 0
		\end{pmatrix}\\
		\boldsymbol{\hat{S}}^{2}&=\dfrac{3\hbar^{2}}{4}\begin{pmatrix}
			1	& 0\\
			0	&1
		\end{pmatrix}
	\end{align}
	\quad\quad 容易验证对易关系
	\begin{equation}
		\overleftrightarrow{\sigma_{\alpha}}\overleftrightarrow{\sigma_{\beta}}=\delta_{\alpha\beta}\overleftrightarrow{I}+\mathrm{i}\epsilon_{\alpha\beta\gamma}\overleftrightarrow{\sigma_{\gamma}}
	\end{equation}
	\section{电磁作用}
	电磁作用的一个经典例子是拉莫尔进动。假设$s=\dfrac{1}{2}$粒子静止在磁场$\boldsymbol{B}=B_{0}\boldsymbol{e_{z}}$中,哈密顿量为
	\begin{equation}
		\overleftrightarrow{H}=-\gamma B_{0}\overleftrightarrow{S_{z}}=-\dfrac{\gamma B_{0}\hbar}{2}\begin{pmatrix}
			1	& 0\\
			0	&-1
		\end{pmatrix}
	\end{equation}
	\quad\quad 只需求解本征值问题
	\begin{equation}
		\overleftrightarrow{H}\chi=E\chi
	\end{equation}
	\quad\quad 很容易解出
	\begin{equation}
		\chi(t)=\begin{pmatrix}
		\cos(\frac{\alpha}{2})\mathrm{e}^{\frac{\mathrm{i}\gamma B_{0}}{2}t}	\\\sin(\frac{\alpha}{2})\mathrm{e}^{-\frac{\mathrm{i}\gamma B_{0}}{2}t}
		\end{pmatrix}
	\end{equation}
	\quad\quad 计算期望值
	\begin{align}
		\left<S_{x}\right>&=\chi^{\dagger}\overleftrightarrow{S_{x}}\chi=\dfrac{\hbar}{2}\sin\alpha\cos\left(\gamma B_{0}t\right)\\
		\left<S_{y}\right>&=\chi^{\dagger}\overleftrightarrow{S_{x}}\chi=-\dfrac{\hbar}{2}\sin\alpha\sin\left(\gamma B_{0}t\right)\\
		\left<S_{z}\right>&=\chi^{\dagger}\overleftrightarrow{S_{x}}\chi=\dfrac{\hbar}{2}\cos\alpha
	\end{align}
	\quad\quad 另一个例子是A-B效应。首先写出电磁场中带电粒子的哈密顿量,由于需要满足对易关系,这里的动量是正则动量
	\begin{equation}
		\hat{H}=\dfrac{1}{2m}\left(\boldsymbol{p}-q\boldsymbol{A}\right)^{2}+q\phi=\dfrac{1}{2m}\left(-\mathrm{i}\hbar\boldsymbol{\nabla}-q\boldsymbol{A}\right)^{2}+q\phi
	\end{equation}
	\quad\quad 将无旋度区域波函数设为
	\begin{equation}
		\Psi=\mathrm{e}^{\mathrm{i}\frac{q}{\hbar}\int^{\boldsymbol{r}}\boldsymbol{A}(\boldsymbol{r^{'}},t)\cdot\mathrm{d}\boldsymbol{r^{'}}}\Psi_{0}
	\end{equation}
	\quad\quad 带入薛定谔方程得
	\begin{equation}
		\mathrm{i}\hbar\dfrac{\partial\Psi}{\partial t}=-\dfrac{\hbar^{2}}{2m}\boldsymbol{\nabla}^{2}\Psi_{0}+q\Psi_{0}\int^{\boldsymbol{r}}\left(\boldsymbol{\nabla}\phi+\dfrac{\partial\boldsymbol{A}}{\partial t}\right)\cdot\mathrm{d}\boldsymbol{r^{'}}
	\end{equation}
	\quad\quad 这部分因子对相位无影响,因此相位差为
	\begin{equation}
		\Delta\varphi=\dfrac{q}{\hbar}\oint_{C^{+}}\boldsymbol{A}(\boldsymbol{r^{'}},t)\cdot\mathrm{d}\boldsymbol{r^{'}}=\dfrac{q\Phi}{\hbar}
	\end{equation}
	\quad\quad 很容易看出来这一结果与规范无关,因为任意规范变换
	\begin{equation}
		\phi\to\phi-\dfrac{\partial\Lambda}{\partial t},\boldsymbol{A}\to\boldsymbol{A}+\boldsymbol{\nabla}\Lambda
	\end{equation}
	\quad\quad 不改变
	\begin{equation}
		\boldsymbol{\nabla}\phi+\dfrac{\partial\boldsymbol{A}}{\partial t}
	\end{equation}
\end{document}