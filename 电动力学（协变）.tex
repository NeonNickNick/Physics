\documentclass[12pt, a4paper, oneside]{ctexart}
\usepackage{amsmath, amsthm, amssymb, graphicx}
\usepackage[bookmarks=true, colorlinks, citecolor=blue, linkcolor=black]{hyperref}
\usepackage{fancyhdr}
\usepackage{abstract}
\usepackage{xcolor}
\usepackage{geometry}
\geometry{a4paper,left=0.6in,right=0.6in,top=1.2in,bottom=1.2in}
\setlength\columnsep{0.5in}
\columnseprule=1pt
\pagestyle{fancy}
\fancyhf{}
\fancyhead[R]{\thepage} 
\fancyhead[C]{电动力学的协变形式} 
% 导言区
\title{\Huge 电动力学的协变形式}
\date{\today}
\setlength{\columnsep}{1in}
\newcommand{\blankspace}{\rule{2cm}{0.15mm}}
\begin{document}
	\maketitle
	历史上正是经典力学与麦克斯韦电磁理论的矛盾导致了狭义相对论的诞生,因此经典电动力学本身就是相对论性的,不存在所谓的“低速近似”。使用协变形式能够更加直观地体现出麦克斯韦电磁理论在相对论变换下的不变性。\par
	我们已经知道真空中的麦克斯韦方程组:
	\begin{equation}
		\begin{aligned}
			\begin{cases}
				\boldsymbol{\nabla} \cdot \boldsymbol{E} = \dfrac{\rho}{\varepsilon_{0}} \\
				\boldsymbol{\nabla} \times \boldsymbol{E} = -\dfrac{\partial \boldsymbol{B}}{\partial t} \\
				\boldsymbol{\nabla} \cdot \boldsymbol{B} = 0 \\
				\boldsymbol{\nabla} \times \boldsymbol{B} = \mu_{0}\boldsymbol{J}+\varepsilon_{0}\mu_{0}\dfrac{\partial \boldsymbol{E}}{\partial t}
			\end{cases}
		\end{aligned}
	\end{equation}
	\quad\quad 额外引入:
	\begin{equation}
		\begin{aligned}
			\begin{cases}
				\boldsymbol{E} = -\boldsymbol{\nabla}\phi-\dfrac{\partial\boldsymbol{A}}{\partial t} \\
				\boldsymbol{B} = \boldsymbol{\nabla}\times\boldsymbol{A}
			\end{cases}
		\end{aligned}
	\end{equation}
	 \quad\quad 现在从最小作用量原理出发构造协变形式的电动力学。在以下推导中假定$A^{\mu}=\left(\phi, \boldsymbol{A}\right)$是四矢量。\par 
	由于要构造一个相对论不变的理论,作用量应该是一个洛伦兹标量,对于电磁场中一个带电量为$q$的粒子,目前能够造出的最简单的一个作用量是:
	\begin{equation}
		S=-mc\int \sqrt{\mathrm{d}x_{\mu}\mathrm{d}x^{\mu}}-q\int A_{\mu}\mathrm{d}x^{\mu}
	\end{equation}
	\quad\quad 直接进行变分:
	\begin{equation}
		\begin{aligned}
		\delta S&=-mc\int \dfrac{\mathrm{d}x_{\mu}\mathrm{d}\delta x^{\mu}}{\sqrt{\mathrm{d}x_{\mu}\mathrm{d}x^{\mu}}}-q\int \left(\delta A_{\mu}\mathrm{d}x^{\mu}+A_{\mu}\mathrm{d}\delta x^{\mu}\right)\\
		&=mc\int \dfrac{\mathrm{d}u_{\mu}}{\mathrm{d}s}\delta x^{\mu}\mathrm{d}s-q\int\left(\mathrm{d}x^{\mu}\delta x^{\nu}\partial_{\nu}A_{\mu}-\delta x^{\mu}\dfrac{\mathrm{d}A_{\mu}}{\mathrm{d}s}\mathrm{d}s\right)\\
		&=mc\int \dfrac{\mathrm{d}u_{\mu}}{\mathrm{d}s}\delta x^{\mu}\mathrm{d}s-q\int\left(u^{\nu}\partial_{\mu}A_{\nu}-\dfrac{\mathrm{d}A_{\mu}}{\mathrm{d}s}\right)\delta x^{\mu}\mathrm{d}s\\
		&=mc\int \dfrac{\mathrm{d}u_{\mu}}{\mathrm{d}s}\delta x^{\mu}\mathrm{d}s-q\int F_{\mu\nu}u^{\nu}\delta x^{\mu}\mathrm{d}s\\
		&=0
		\end{aligned}
	\end{equation}
	\quad\quad 以上定义了四速度$u_{\mu}=\dfrac{\mathrm{d}x_{\mu}}{\mathrm{d}s}$、电磁场张量$F_{\mu\nu}=\partial_{\mu}A_{\nu}-\partial_{\nu}A_{\mu}$,由此表示出运动方程为:
	\begin{equation}
		mc\dfrac{\mathrm{d}u_{\mu}}{\mathrm{d}s}=qF_{\mu\nu}u^{\nu}
	\end{equation}
	\quad\quad 可以验证它对应于三维情形下的洛伦兹力公式。电磁场张量可以显式地写为:
	\[
	\renewcommand{\arraystretch}{1.5}
	F_{\mu\nu} =
	\begin{bmatrix}
		0 & \dfrac{E_1}{c} & \dfrac{E_2}{c} & \dfrac{E_3}{c} \\
		-\dfrac{E_1}{c} & 0 & -B_3 & B_2 \\
		-\dfrac{E_2}{c} & B_3 & 0 & -B_1 \\
		-\dfrac{E_3}{c} & -B_2 & B_1 & 0
	\end{bmatrix}
	\]
	\quad\quad 它是一个四阶反对称张量。这说明它满足轮换对称式:
	\begin{equation}
		\partial_{\sigma}F_{\mu\nu}+\partial_{\mu}F_{\nu\sigma}+\partial_{\nu}F_{\sigma\mu}=0
	\end{equation}
	\quad\quad 此前已经知道经典电动力学是规范不变的,这意味着$A_{\mu}$存在额外非物理的自由度,换言之,容易验证规范变换
	\begin{equation}
		A_{\mu}\to A_{\mu}+\partial_{\mu}\Lambda
	\end{equation}
	不改变此前的所有结果。现在的问题是,规范不变性对应的守恒量是什么。在此之前先来考虑如何构造电磁场的作用量。这一作用量必然有一部分来自于电磁场自身,另一部分来自于电磁场与源的相互作用。刚才考虑的点粒子就是一种特殊情况,将它的作用量的第二项改写为:
	\begin{equation}
		-\int \mathrm{d}^{4}x\textcolor{blue}{q\delta^{3}({\boldsymbol{r}-\boldsymbol{r}_{e}(\tau)})}A_{\mu}u^{\mu}
	\end{equation}
	\quad\quad 蓝色的部分实际上是电荷密度。因此定义四电流密度$J^{\mu}=\rho u^{\mu}$,现在可以将电磁场作用量构造为:
	\begin{equation}
		S=-\dfrac{1}{4\mu_{0}}\int \mathrm{d}^{4}xF_{\mu\nu}F^{\mu\nu}-\int \mathrm{d}^{4}xA_{\mu}J^{\mu}
	\end{equation}
	\quad\quad 这里$\mathrm{d}^{4}x$是四维体元,由于洛伦兹变换矩阵行列式为$1$,它也是一个洛伦兹不变量。现在来检查这个作用量是否具有规范不变性。只需要检查第二项,考虑到:
	\begin{equation}
		\int\mathrm{d}^4xJ^{\mu}\partial_{\mu}\Lambda\to\int\mathrm{d}^4x\Lambda\partial_{\mu}J^{\mu}
	\end{equation}
	\quad\quad 因此规范不变性对应于电荷守恒$\partial_{\mu}J^{\mu}=0$。现在对$A_{\mu}$进行变分:
	\begin{equation}
		\begin{aligned}
			\delta S&=-\dfrac{1}{2\mu_{0}}\int \mathrm{d}^{4}xF^{\mu\nu}\delta F_{\mu\nu}-\int \mathrm{d}^{4}xJ^{\mu}\delta A_{\mu}\\
			&=-\dfrac{1}{2\mu_{0}}\int \mathrm{d}^{4}xF^{\mu\nu}\left(\partial_{\mu}\delta A_{\nu}-\partial_{\nu}\delta A_{\mu}\right)-\int \mathrm{d}^{4}xJ^{\nu}\delta A_{\nu}\\
			&=-\dfrac{1}{\mu_{0}}\int \mathrm{d}^{4}xF^{\mu\nu}\partial_{\mu}\delta A_{\nu}-\int \mathrm{d}^{4}xJ^{\nu}\delta A_{\nu}\\
			&=\dfrac{1}{\mu_{0}}\int \mathrm{d}^{4}x\delta A_{\nu}\partial_{\mu}F^{\mu\nu}-\int \mathrm{d}^{4}xJ^{\nu}\delta A_{\nu}\\
			&=0
		\end{aligned}
	\end{equation}
	\quad\quad 于是得到了协变形式的麦克斯韦方程组,可以验证它与三维形式具有如下对应关系:
	\begin{equation}
		\begin{aligned}
		\begin{cases}\partial_{\sigma}F_{\mu\nu}+\partial_{\mu}F_{\nu\sigma}+\partial_{\nu}F_{\sigma\mu}=0\begin{cases}
				\boldsymbol{\nabla} \times \boldsymbol{E} = -\dfrac{\partial \boldsymbol{B}}{\partial t} \\
				\boldsymbol{\nabla} \cdot \boldsymbol{B} = 0 
				
			\end{cases}
			\\\partial_{\mu}F^{\mu\nu}=\mu_{0}J^{\nu}\begin{cases}\boldsymbol{\nabla} \cdot \boldsymbol{E} = \dfrac{\rho}{\varepsilon_{0}} \\
				\boldsymbol{\nabla} \times \boldsymbol{B} = \mu_{0}\boldsymbol{J}+\varepsilon_{0}\mu_{0}\dfrac{\partial \boldsymbol{E}}{\partial t}
				
			\end{cases}
			
		\end{cases}
		\end{aligned}
	\end{equation}
	\quad\quad 从以上推导过程可以看出,电磁场的拉格朗日量密度可写为$\mathcal{L}=\mathcal{L}(A_{\mu},\partial_{\nu}A_{\mu},t)$,于是可以直接变分:
	\begin{equation}
		\begin{aligned}
			\delta S&=\int \mathrm{d}^{4}x\delta\mathcal{L}(A_{\mu},\partial_{\nu}A_{\mu},t)\\
			&=\int\mathrm{d}^{4}x\dfrac{\partial\mathcal{L}}{\partial A_{\mu}}\delta A_{\mu}+\int\mathrm{d}^{4}x\dfrac{\partial\mathcal{L}}{\partial\left(\partial_{\nu}A_{\mu}\right)}\partial_{\nu}\delta A_{\mu}\\
			&=\int\mathrm{d}^{4}x\delta A_{\mu}\dfrac{\partial\mathcal{L}}{\partial A_{\mu}}-\int\mathrm{d}^{4}x\delta A_{\mu}\partial_{\nu}\dfrac{\partial\mathcal{L}}{\partial\left(\partial_{\nu}A_{\mu}\right)}\\
			&=0\to\partial_{\nu}\dfrac{\partial\mathcal{L}}{\partial\left(\partial_{\nu}A_{\mu}\right)}=\dfrac{\partial\mathcal{L}}{\partial A_{\mu}}
		\end{aligned}
	\end{equation}
	\quad\quad 现在考虑无源情形下的的无穷小变换变换:
	\begin{equation}
		A_{\mu}\to A_{\mu}+\epsilon_{\mu}
	\end{equation}
	\quad\quad 拉格朗日量密度变化:
	\begin{equation}
		\begin{aligned}
			\delta\mathcal{L}&=\dfrac{\partial\mathcal{L}}{\partial A_{\mu}}\epsilon_{\mu}+\dfrac{\partial\mathcal{L}}{\partial\left(\partial_{\nu}A_{\mu}\right)}\partial_{\nu}\epsilon_{\mu}\\
			&=\partial_{\nu}\left(\dfrac{\partial\mathcal{L}}{\partial\left(\partial_{\nu}A_{\mu}\right)}\right)\epsilon_{\mu}+\dfrac{\partial\mathcal{L}}{\partial\left(\partial_{\nu}A_{\mu}\right)}\partial_{\nu}\epsilon_{\mu}\\
			&=\partial_{\nu}\left(\dfrac{\partial\mathcal{L}}{\partial\left(\partial_{\nu}A_{\mu}\right)}\epsilon_{\mu}\right)\\
			&=\eta_{\mu\nu}\epsilon_{\mu}\partial^{\mu}\mathcal{L}
		\end{aligned}
	\end{equation}
	\quad\quad 定义:
	\begin{equation}
	\begin{aligned}
		\Theta^{\nu \rho}=\frac{\partial \mathcal{L}}{\partial\left(\partial_{\nu} A^{\mu}\right)} \partial^{\rho} A^{\mu}-\eta^{\nu \rho} \mathcal{L}&=-\frac{1}{\mu_{0}} F^{\nu \mu} \partial^{\rho} A_{\mu}+\frac{1}{4 \mu_{0}} \eta^{\nu \rho} F^{\mu \sigma} F_{\mu \sigma}\\
		\partial_{\nu} \Theta^{\nu \rho}&=0
	\end{aligned}
	\end{equation}
	\quad\quad 美中不足的是上式不是规范不变的。可以利用$\partial_{\nu}\left(F^{\nu\mu}\partial_{\mu}A^{\rho}\right)=\partial_{\nu}\partial_{\mu}\left(F^{\nu\mu}A^{\rho}\right)=0$,重新定义电磁场能动张量:
	\begin{equation}
		\begin{aligned}
			T^{\nu \rho}=-\frac{1}{\mu_{0}}\left( F^{\nu \mu} \partial^{\rho} A_{\mu}-F^{\nu\mu}\partial_{\mu}A^{\rho}\right)+\frac{1}{4 \mu_{0}} &\eta^{\nu \rho} F^{\mu \sigma} F_{\mu \sigma}=\frac{1}{\mu_{0}} F^{\nu \mu} F_{\mu}^{\rho}+\frac{1}{4 \mu_{0}} \eta^{\nu \rho} F^{\mu \sigma} F_{\mu \sigma}\\
			\partial_{\nu} T^{\nu \rho}=&0
		\end{aligned}
	\end{equation}
	\quad\quad 在真空无源情形下,能动张量是一个四阶对称张量,可显式地写为:
	\[
	\renewcommand{\arraystretch}{1.3}
	\overleftrightarrow{T}  =
	\begin{bmatrix}
		u & \dfrac{\boldsymbol{S}}{c}\\
		c\boldsymbol{g} & \overleftrightarrow{\sigma}\\
		
	\end{bmatrix}
	\]
	\quad\quad 其中$u,\boldsymbol{S},\boldsymbol{g},\overleftrightarrow{\sigma}$分别为电磁能量密度、坡印廷矢量、电磁动量密度、麦克斯韦电磁应力张量:
	\begin{equation}
		\begin{aligned}
			\begin{cases}
				u = \dfrac{1}{2}\varepsilon_{0}\boldsymbol{E}^{2}+\dfrac{1}{2\mu_{0}}\boldsymbol{B}^{2} \\
				\boldsymbol{S} = \dfrac{1}{\mu_{0}}\boldsymbol{E}\times\boldsymbol{B} \\
				\boldsymbol{g} = \varepsilon_{0}\boldsymbol{E}\times\boldsymbol{B} \\
				\overleftrightarrow{\sigma}= \left(\dfrac{1}{2}\varepsilon_{0}\boldsymbol{E}^{2}+\dfrac{1}{2\mu_{0}}\boldsymbol{B}^{2}\right)\overleftrightarrow{I}-\overleftrightarrow{\left(\varepsilon_{0}\boldsymbol{E}\boldsymbol{E}+\dfrac{1}{\mu_{0}}\boldsymbol{B}\boldsymbol{B}\right)}
			\end{cases}
		\end{aligned}
	\end{equation}
	\quad\quad 以上讨论的电磁场方程成立的一个前提条件是光子是零质量的。可以用$Proca$理论描述有质量类似光子的粒子,它不改变电磁场张量的定义,将自由电磁场拉格朗日量密度改写为:
	\begin{equation}
		\mathcal{L}=-\dfrac{1}{4\mu_{0}}F_{\mu\nu}F^{\mu\nu}+\dfrac{1}{2}m^{2}A_{\mu}A^{\mu}
	\end{equation}
	\quad\quad 此时的电磁场含源方程变为了:
	\begin{equation}
		\partial_{\mu}F^{\mu\nu}+m^{2}A^{\nu}=\mu_{0}J^{\nu}
	\end{equation}
	\quad\quad 上式强制要求:
	\begin{equation}
		m^{2}\partial_{\nu}A^{\nu}=\mu_{0}\partial_{\nu}J^{\nu}-\partial_{\nu}\partial_{\mu}F^{\mu\nu}=0
	\end{equation}
	\quad\quad 这种电磁场理论没有规范对称性。实际上规范对称性要求光子质量为$0$。进一步利用这一结果化简含源方程:
	\begin{equation}
		(\partial_{\mu}\partial^{\mu}+m^{2})A^{\nu}=\mu_{0}J^{\nu}
	\end{equation}
	\quad\quad 实际上这是克莱因-戈登方程,它的色散关系与经典电磁波不同。如果考虑静场,上式可拆分为:
	\begin{equation}
		\begin{aligned}
			\begin{cases}
				(\boldsymbol{\nabla}^{2}-m^{2})\phi = -\dfrac{\rho}{\varepsilon_{0}} \\
				(\boldsymbol{\nabla}^{2}-m^{2})\boldsymbol{A} = -\mu_{0}\boldsymbol{J}
			\end{cases}
		\end{aligned}
	\end{equation}\quad\quad
	它具有积分形式解:
	\begin{equation}
		\begin{aligned}
			\begin{cases}
				\phi(\boldsymbol{r^{'}}) = \dfrac{1}{4\pi\varepsilon_{0}}\displaystyle\int\mathrm{d}^{3}x^{'}\dfrac{\rho(\boldsymbol{r^{'}})}{\left|\boldsymbol{r}-\boldsymbol{r^{'}}\right|}\mathrm{e}^{-m\left|\boldsymbol{r}-\boldsymbol{r^{'}}\right|} \\
				\boldsymbol{A}(\boldsymbol{r^{'}}) = \dfrac{\mu_{0}}{4\pi}\displaystyle\int\mathrm{d}^{3}x^{'}\dfrac{\boldsymbol{J}(\boldsymbol{r^{'}})}{\left|\boldsymbol{r}-\boldsymbol{r^{'}}\right|}\mathrm{e}^{-m\left|\boldsymbol{r}-\boldsymbol{r^{'}}\right|}
			\end{cases}
		\end{aligned}
	\end{equation}
	\quad\quad 在质量为$0$时直接就回到经典形式。
\end{document}